\textbf{Hilbert space ($\mathcal{H}$) and vector space:}

A Hilbert space is a complex or real vector space in infinite dimensions, provided with the ability to define an inner product. 
The mathematical definition of a Vector space includes the following properties:
\begin{itemize}
    \item A set of vectors in the Vector space can be added together
    and still remain in the vector space, $\Psi, \Phi \in V \text{ then } \Psi+\Phi \in V$.
    \item Vectors can be multiplied by scalars and still remain in the vector space
\end{itemize}
So a vector space can include functions, matricies, vectors, etc.

The definition of a Hilbert space adds the following properties to the vector space:
\begin{itemize}
    \item An inner product is defined, $\langle \Psi | \Phi \rangle$ which maps two vectors to a scalar (the inner product is just a number). 
    The inner product satisfies the following properties:
    \begin{itemize}
        \item Conjugate symmetry: $\langle \Psi | \Phi \rangle = \langle \Phi | \Psi \rangle^* = \overline{\langle \Phi | \Psi \rangle}$
        \item Linearity in the second argument: $\langle \Psi | a\Phi_1 + b\Phi_2 \rangle = a\langle \Psi | \Phi_1 \rangle + b\langle \Psi | \Phi_2 \rangle$
        \item Positive-definiteness: $\langle \Psi | \Psi \rangle \geq 0$ with equality if and only if $\Psi = 0$
    \end{itemize}
    \item The space is complete, meaning that all Cauchy sequences converge to a limit within the space. 
    \begin{itemize}
        \item A Cauchy sequence is a sequence of elements who become arbitrarily close to each 
        other as the sequence progresses. 
        Formally, a sequence $\{x_n\}$ is Cauchy if for every $\epsilon > 0$, 
        there exists an integer $N$ such that for all $m, n > N$, 
        the distance between $x_m$ and $x_n$ is less than $\epsilon$:
    \end{itemize} 
\end{itemize}

\begin{figure}[H]
  \centering
  \begin{subfigure}{0.5\textwidth}
    \centering
    \includegraphics[width=\textwidth]{Figures/vector_space.png}
    \caption{Vector space axoims.}
    \label{fig:vector space}
  \end{subfigure}
  \hfill
  \begin{subfigure}{0.25\textwidth}
    \centering
    \includegraphics[width=\textwidth]{Figures/cauchy_sequence.png}
    \caption{Cauchy sequence converging to a limit within the space.}\label{cauchy sequence}
  \end{subfigure}
\end{figure}


So in summary, a Hilbert space is a inner product space that is complete. 
In QM a Hilbert space is used to describe the state space of a quantum system, 
where each point in the Hilbert space corresponds to a possible quantum state of the system.
A quantum state is represented by a state vector (all vectors in the space, called ket) where we have the properties described above 
as $| \Psi \rangle = | \psi_1 \rangle + | \psi_2 \rangle$, $| \psi \rangle \cdot c = c\cdot | \psi \rangle  $ 
and for $c=0$ the ket is called a null ket. An important postulate in QM is that for $c \neq 0$ then $| \alpha \rangle \text{ and } c | \alpha \rangle$ are the same physical state.
This means that only the direction of the ket matters, not the magnitude.
A quantum state can be expressed as a linear combination of basis states 
(the set of kets where for every state vector they can be written as a linear combination of them (bases states)) in the Hilbert space,
which allows for the superposition principle in quantum mechanics (any let in the space
can be written as $|\alpha \rangle = \sum_{a'} c_{a'} |a'\rangle$ with $c_{a'}$ as a complex coefficient). \textcolor{red}{maybe talk about orthogonal and orthonormal}
In addition the inner product must be normalized to 1,
$\langle \Psi | \Psi \rangle = 1$, to ensure that the total probability of all possible outcomes of a measurement is 1.

Example: 
Basis states: $\{ | 0 \rangle, |1\rangle \}$, with $| \Psi \rangle = c_1 |0\rangle + c_2 |1\rangle \in \mathcal{H}$. Here $c_1, c_2$ are just factors.





\textbf{Dual space: }
The dual space of a Hilbert space $\mathcal{H}$, denoted as $\mathcal{H}^*$,
 is the set of all linear functionals that map vectors from the 
Hilbert space to the corresponding complex Conjugated space of the Hilbert space (usually the complex numbers $\mathbb{C}$ or real numbers $\mathbb{R}$).


The dual space is the space containing the bra vectors $\langle \Psi |$ corresponding to the ket vectors $| \Psi \rangle$ in the Hilbert space.
 So we have $ |\alpha \rangle \stackrel{DC}{\longleftrightarrow} \langle \alpha |$ $\Rightarrow$
$c_{\alpha} | \alpha \rangle + c_{\beta} |\beta \rangle \stackrel{DC}{\longleftrightarrow} c_{\alpha}^{\star} \langle \alpha |+ c_{\beta}^{\star} \langle\beta|$. Here DC 
means dual correspondence. 
\begin{figure}[H]
  \centering
  \includegraphics[width=0.2\textwidth]{Figures/dual_hilbert_space.png}
  \caption{}\label{dual hilbert space}
\end{figure}
The dual space is important in quantum mechanics because it allows us to define inner products ($\langle \alpha |\beta \rangle$), which is just a 
reciepe to integrate.
For the inner product the properties are described above in the Hilbert space section. If the inner product is zero, the kets are orthogonal.
In addition if they are orthogonal and normalized they are orthonormal 
($| \tilde{\alpha} \rangle= \frac{1}{\sqrt{\alpha \cdot \alpha }} | \alpha \rangle$ where $\langle  \tilde{\alpha}| \tilde{\alpha}\rangle = 1$) ($\sqrt{\alpha \cdot \alpha }$ is the norm of the ket).
Since magnitude has no physical meaning in QM, we often work with orthonormal basis states.


\textbf{Operators acting on states:}

In quantum mechanics, operators are mathematical objects that act on the state vectors (kets) in a Hilbert 
space to produce new state vectors or to extract information about physical observables.

In QM an observable (like position, momentum, energy) is represented by an operator $\hat{A}$.
When an operator acts on a state vector, it transforms the state into another state vector. The 
operator acts on the ket from the left side, $\hat{A} | \alpha \rangle $ and on a bra from the right, $\langle \alpha | \hat{A}$. Thereby, is $\hat{A}$
not a constant, but a function that takes a ket as input and produces another ket as output. But if the ket
is an eigenket of the operator (e.g. $| a' \rangle , | a'' \rangle, ... $), then the output is just a scaled version of the input ket:
\begin{equation}
    \hat{A} | a' \rangle = a' | a' \rangle, \quad \hat{A} | a'' \rangle = a'' | a'' \rangle, \quad ...
\end{equation}
where $a', a'', ...$ are the eigenvalues (just numbers) corresponding to the eigenkets.The physical state of the eigenket 
is called an eigenstate. 
\textcolor{red}{maybe example with spin operators, eqn. 1.20}
We define: $ \hat{A}|\alpha \rangle \stackrel{DC}{\longleftrightarrow} \langle \alpha | \hat{A}^{\dagger}$. 
Here $\hat{A}^{\dagger}$ is the Hermitian adjoint operator of $\hat{A}$, which is defined such that $\langle \alpha | \hat{A} | \beta \rangle = \langle \beta | \hat{A}^{\dagger} | \alpha \rangle^{\star}$.
Some rules applies: $\hat{A}\hat{B}\neq \hat{B}\hat{A}$, $\left( \hat{A}\hat{B} \right)^{\dagger} = \hat{B}^{\dagger}\hat{A}^{\dagger}$.

We can define the outer product of a ket and a bra as $\hat{A} = | \alpha \rangle \langle \beta |$ which is an operator.

Operators can be classified into different types based on their properties. Some common types of operators in quantum mechanics include:
\begin{itemize}
    \item Hermitian operators: These operators are equal to their own Hermitian adjoint ($\hat{A} = \hat{A}^{\dagger}$) and represent physical observables. 
    Their eigenvalues are real numbers, which correspond to the possible measurement outcomes of the observable. Their 
    eigenkets form an orthonormal basis for the Hilbert space.
    \item Unitary operators: These operators preserve the inner product between state vectors and represent time evolution and symmetry transformations in quantum mechanics,
     \[\hat{U}\hat{U}^{\dagger} = \hat{U}^{\dagger}\hat{U}= \hat{I}\].
    \item Projection operators: These operators project state vectors onto specific subspaces of the Hilbert space, often used to describe measurements.
    \item Ladder operators: These operators are used to raise or lower the quantum numbers of certain states, such as in the case of the quantum harmonic oscillator or angular momentum.
\end{itemize}
\textcolor{red}{Projection operator, i do not think we have worked with that!!!! so maybe delete if i do not find anything with it when reviewing the qurriculum}. 
Illegal products: $ |\alpha \rangle \hat{A}$ and $\hat{A} \langle \alpha |$.

I now want to show the associative axiom for operators acting on kets:
\begin{align*}
    &1. \text{ Associative/incompatible: }\hat{A}(\hat{B} | \alpha \rangle) = (\hat{A}\hat{B}) | \alpha \rangle\\
    &2. \text{ Distributive: } \hat{A} (c_1 | \alpha_1 \rangle + c_2 | \alpha_2 \rangle) = c_1 \hat{A} | \alpha_1 \rangle + c_2 \hat{A} | \alpha_2 \rangle\\
    &3. \text{ Scalar multiplication: } \hat{A} (c | \alpha \rangle) = c (\hat{A} | \alpha \rangle)\\
    &4. \text{ Non-cummutive}: \hat{A}\hat{B}\neq \hat{B}\hat{A}.
\end{align*}
These properties ensure that operators behave consistently when applied to state vectors in a Hilbert space.
These also lead to $(|\beta \rangle \langle \alpha |) \cdot|\gamma \rangle = |\beta \rangle \cdot (\langle \alpha | \gamma \rangle$. 
So here the prove that an outer product acting on a ket gives a ket again $\Rightarrow$ operator. On the other hand if 
$(\langle \alpha | \gamma \rangle) \cdot | \beta \rangle$ this would not apply, because then we would have an illegal product bra acting on ket.
The second property applies for operators that are linear operators which ensures superposition. These are the operators we work with un QM with the exeption of time evolution operators. 
The expectation value of an operator $\hat{A}$ in a state $| \alpha \rangle$ is given by $\langle \hat{A} \rangle = \langle \alpha | \hat{A} | \alpha \rangle$.

\textbf{Operators with a continuous spectrum:}

In quantum mechanics, operators with a continuous spectrum are those whose eigenvalues 
form a continuous infitnite range of possible outcomes rather than discrete values (e.g. spin $\hat{S}_z$: $\frac{1}{2}$ or $-\frac{1}{2}$).
An example of such an operator is the momentum operator $\hat{p}$ in one dimension, which has eigenvalues that can take any real value from $-\infty$ to $\infty$.
Operators with a continuous spectrum has the same properties as operators with a discrete spectrum, but the treatment of their eigenvalues and eigenstates requires the use of integrals instead of sums.
In addition the eigenstates are, instead of a Kronecker delta,  represented by a Dirac delta function ($\langle a' | a'' \rangle = \delta_{a',a''} \rightarrow \langle \xi' | \xi'' \rangle = \delta(\xi' - \xi'')$).

\underline{Momentum operator and its relation to translations:}

As said, the momentum operator is an example of an operator with a continuous spectrum.
In one dimension, the momentum operator is defined as: $\hat{p} = -i\hbar \frac{\partial}{\partial x}$. 
However now I want to discuss where it comes from, by looking at translations in space.
The momentum operator is closely related to the concept of translations in space.

The translation operator $\hat{T}(a)$ is defined as a desplacement operator. E.g. it takes a state $| \psi(x) \rangle$ and translates it by a distance $a$ without changing anything about it. 
The translation operator has to be unitary to preserve the inner product and probabilities in QM.
It is a infiniteimal translation defined as: 
\begin{equation}
    \hat{T}(dx') | x' \rangle = | x'+dx' \rangle. 
\end{equation}
So we end up with a position eigenket with eigenvalue $x' + dx'$. \textcolor{red}{Should i prove the properties for T????}
The following properties applies to the translation operator:
\begin{itemize} 
    \item It is unitary: $\hat{T}^{\dagger}(a) \hat{T}(a) = \hat{T}(a) \hat{T}^{\dagger}(a) = \hat{I}$
    \item It has the composition property: $\hat{T}(a) \hat{T}(b) = \hat{T}(a+b)$
    \item The inverse translation is given by: $\hat{T}^{-1}(a) = \hat{T}(-a)$
    \item The translation operator is non Hermitian: $\hat{T}^{\dagger}(a) \neq \hat{T}(a)$.
\end{itemize}
We now try to express the translation operator in terms of a Hermitian operator $\hat{K}$ as:
\begin{equation}
    \hat{T}(dx') = \hat{I} - i\hat{K}\cdot  dx'    
\end{equation}
where $\hat{I}$ is the identity operator. By accepting this form we can derive the fundamental commutation relation $[x_i , K_j]=i\delta_{ij}$. 

The commutation relation can be derived by comsidering the action of the translation operator on a position eigenket and a 
position operator $\hat{x}$.
We have that $\hat{T}(dx') \hat{x} | x' \rangle = x' | x' + dx' \rangle$ and $\hat{x} \hat{T}(dx') | x' \rangle = (x'+dx') | x' + dx' \rangle$.
By subtracting these two equations we get
\begin{equation}
    [\hat{x}, \hat{T}(dx')] | x' \rangle = dx' | x' + dx' \rangle.
\end{equation}
By inserting the expression for $\hat{T}(dx')$ we get:  
\begin{equation}
    [\hat{x}_i, \hat{K}_j] =i \delta_{ij}  .
\end{equation}

By using this we can identify $\hat{K}$ with the momentum operator divided by $\hbar$, $\hat{K} = \frac{\hat{p}}{\hbar}$ \textcolor{red}{How is this right???? In addition I do not understand the first eqn. on page 235 in Griffths}.
From this I get
\begin{equation}
    \hat{T}(dx') = \hat{I} - i\frac{p}{\hbar}\cdot  dx'.
\end{equation} 
This also makes sense because the momentum operator is basically just an operator that describes what happens to the state as it is shifted by a small displacement. 
\textit{Therefore the momentum operator is the generator of translations in space.}
So the K operator can be expressed in other terms by using the de Broglie relation $p = \hbar k$ where $k$ is the wave number. 
This gives that k (wave number) is the eigenvalue of the operator $\hat{K}$, $\hat{K} | k \rangle = k | k \rangle$. The K operator can be called the wave-number operator. 

By looking at a finitite translation $\Delta x'$ and let the translation happen N times, we then obtain (as $N\rightarrow \infty$): 
\begin{align*}
    \hat{T}(dx)=e^{-\frac{i}{\hbar} \hat{p} dx'}.
\end{align*}



Eigenfunctions: plane waves $ \langle x | p \rangle = \frac{1}{\sqrt{2\pi \hbar}} e^{ipx/\hbar}$