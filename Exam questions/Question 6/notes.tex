\textbf{Relativistic QM:}

The Schrödinger equation only works for non -relativistic particles. 
This is due to the fact that it is derived under the assumption that the kinetic energy of the particle is much less than its rest mass 
energy (i.e., \( E \ll mc^2 \)). This arises from the fact that the SE are based on a classical expression for the Hamiltonian ($T=p^2/2m$), where 
in Relativity, the total energy of a particle is given by \( E^2 = p^2c^2 + m^2c^4 \).
So in order to accurately describe particles moving at speeds close to the speed of light (e.g. electrons), we need to use relativistic quantum mechanics.
So we need a new kind of relaticistic wave equation.

Two of the most important relativistic wave equations are the Klein-Gordon equation and the Dirac equation.

\underline{Klein-Gordon (KG) eqn.:}

The Klein-Gordon eqn. was derived by the same approach as for the SE, but starting from the relativistic energy-momentum relation $E^2 = p^2c^2 + m^2c^4 $ instead.
This equation is given by:
\begin{align*}
    \left( \frac{1}{c^2} \frac{\partial^2}{\partial t^2} - \nabla^2 + \frac{m^2c^2}{\hbar^2} \right) \psi(\mathbf{r},t) = 0
\end{align*}
However the KG eqn. has some problems. It includes second-order time derivatives, which leads to issueas when 
determing the initial conditions for the wave function. Specifically, both the initial wave function and its first time derivative must be specified,
which is different from the SE where only the initial wave function is needed. Therefore, the WF itsetf does not provide a complete description of the systems quantum states.
Therefore the KG eqn. is inconsitent with one of the central principles of quantum mechanics, which states that the wave function should fully describe the physical system.
In addition to this, the KG eqn. has other issues when it comes to its solutions and interpretation:
\begin{itemize}
    \item The probability density associated with the KG eqn. is not positive definite, 
    which means that it can take on negative values. 
    This is problematic because probabilities must always be non-negative.
    \item The KG eqn. allows for solutions with negative energy, 
    which leads to issues with the interpretation of the wave function and the 
    stability of the vacuum.
\end{itemize}
The negative energy solutions arises from the solution given by $\psi(\mathbf{r}, t) = N e^{\frac{i}{\hbar}(p\cdot x -Et)}$. 
If this is substituted into the KG eqn., we get the relativistic energy-momentum relation, which has both positive and negative energy solutions:
\begin{align*}
    E^2 \psi = (p^2c^2 + m^2c^4) \psi \implies
    E = \pm \sqrt{p^2c^2 + m^2c^4}
\end{align*}
This means that for every positive energy solution, there is a corresponding negative energy solution.
If negative energy states were allowed in a single particle interpretation, a particle could 
decay into lower and lower energy states without bound, and thereby emmitting an infinite amount of energy - which is in odd with physical observations.
It therefore also challenges the stability of the vacuum, as particles could spontaneously transition to these negative energy states with no lower limit, and thereby no ground state. 
The probability density issue arises from the fact that the probability density for the KG eqn. is given by:
\begin{align*}
    \rho(\mathbf{x}, t) &= \frac{i\hbar}{2mc^2} \left( \psi^* \frac{\partial \psi}{\partial t} - \psi \frac{\partial \psi^*}{\partial t} \right)\\
    &= \frac{E}{mc^2} |\psi|^2 \quad \text{for plane wave solutions: } \psi(\mathbf{r}, t) = N e^{\frac{i}{\hbar}(p\cdot x -Et)}
\end{align*}
This expression can take on negative values, which is problematic for a probability density. This is 
due to the possibility of negative energy solutions, which can lead to negative values for \( \rho \).
This undermines the interpretation of the wave function as a probability amplitude.
Due to these issues, the KG eqn. is not suitable for describing single particles in a relativistic context. 
It overlooks key features such as spin and fine structure. And it fails to predict the correct energy levels of the hydrogen atom.


\underline{Dirac eqn.:}

Dirac sought to develop a relativistic wave equation that would overcome the limitations of the KG eqn. and that would 
be first-order in both space and time derivatives. This would allow for a more consistent interpretation of the wave function.
He wanted the eqn. to be fully consistent with special relativity, so it had to be 
both linear in the space and time derivatives, and it had to respect the principles of Lorentz invariance, of momentum and energy, 
$E^2 =p^2c^2 + m^2 c^4$. If we start from the relativistic energy-momentum relation, we can write:
$E= \sqrt{p^2c^2 + m^2 c^4}$ and then get $\sqrt{p^2c^2 + m^2 c^4}\psi$, If we then apply $E\rightarrow i\hbar \frac{\partial}{\partial t}$
we get $i \hbar  \frac{\partial}{\partial t} \psi= \sqrt{p_x^2c^2 + p_y^2 c^2 + p_z^2 c^2 m^2 c^4 }\psi$. 
However, the square root operator is non-local and difficult to work with.
So he wanted to reqwrite the equation in a linear form involving $p_x, p_y, p_z$. 
To achieve this, Dirac proposed an equation of the form:
\begin{align*}
    &i \hbar \frac{\partial}{\partial t} \psi = \left( c \alpha_x p_x + c \alpha_y p_y + c \alpha_z p_z + \beta mc^2 \right) \psi \\
    &\text{where } \alpha_x, \alpha_y, \alpha_z, \beta \text{ are matrices that need to be determined.}\\
    &\vec{p}^2c^2 + m^2 c^4 = (c \alpha_x p_x + c \alpha_y p_y + c \alpha_z p_z + \beta mc^2)^2.    
\end{align*}
For this to be true we must have $\alpha_x^2 = \alpha_y^2 = \alpha_z^2 = \beta^2 = 1$ and for the cross terms to vanish, we must have
$\{\alpha_i, \alpha_j\} = 0$ for $i \neq j$ and $\{\alpha_i, \beta\} = 0$. In order for these 
relations to holds the $\alpha_i$ and $\beta$ must be represented by matrices, specifically $4 \times 4$ matrices.
If we choose to look at the pauli $2\times 2$ matricies we get that $\alpha_i=\sigma_i$ and $\beta=0$ but we know that $\beta^2=1$ so this cannot be true.
Therefore we need to use $4\times 4$ matrices. If we choose to again look at the pauli matrices, we can construct the following representation:
\begin{align*}
    \alpha_i = \begin{pmatrix}
        0 & \sigma_i \\
        \sigma_i & 0
    \end{pmatrix}, \quad
    \beta = \begin{pmatrix}
        I & 0 \\
        0 & -I
    \end{pmatrix}
\end{align*}
Where \( I \) is the \( 2 \times 2 \) identity matrix. These matricies satisfy the required anticommutation relations. 
We now have the Dirac equation in the form:
\begin{align*}
    i \hbar \frac{\partial}{\partial t} \psi = \left( -i \hbar c \vec{\alpha} \cdot \nabla + \beta mc^2 \right) \psi
\end{align*}
Here $\alpha = \begin{bmatrix}
    \alpha_x \\
    \alpha_y \\ 
    \alpha_z
\end{bmatrix}$.
This can be rewritten as:
\begin{align*}
    (i\hbar \gamma^{\mu} \partial_{\mu} - mc) \psi = 0.
\end{align*}
This is the covariant form of the Dirac eqn. because it is Lorentz invariant and its structure is preserved under Lorentz transformations.
Here $\gamma^0= \beta, \gamma^1 = \beta \alpha_x, \gamma^2 = \beta \alpha_y, \gamma^3 = \beta \alpha_z$ and 
$\psi(x,t) = \begin{bmatrix}
    \psi_1(x,t) \\
    \psi_2(x,t) \\
    \psi_3(x,t) \\
    \psi_4(x,t)
\end{bmatrix}$ which is a four-component spinor wave function (Dirac spinor). 
Here each components are complex functions that describe the quantum state of the particle. In this 
case when calculating the probability density we get $\rho = \mathbf{\psi}^{\dagger} \mathbf{\psi}$, which is always positive definite sinced the 
modulus squared of a complex number is always equal to or greater than zero, and the 
probability current is given by $\mathbf{j} = c \mathbf{\psi}^{\dagger} \alpha \mathbf{\psi}$.


\textbf{A free particle:}

We here look at a free particle solution to the Dirac eqn. of the form:
\begin{align*}
    \psi(\mathbf{r}, t) = u(\mathbf{p}) e^{\frac{i}{\hbar}(\mathbf{p} \cdot \mathbf{r} - Et)}
\end{align*}
where \( u(\mathbf{p}) \) is a four-component spinor that depends on the momentum \( \mathbf{p} \) but not on time or space.
Substituting this into the Dirac equation gives:
\begin{align*}
    E u(\mathbf{p}) = (c \alpha \cdot \mathbf{p} - \beta mc^2)
\end{align*}
which is known as the time independent Dirac equation in momentum space. 
We now look at a stationary particle with \( \mathbf{p} = 0 \). In this case the equation reduces to:
\begin{align*}
    E u(0) = \beta mc^2 u(0)
\end{align*}
The eigenvalues of \( \beta \) are \( +1 \) and \( -1 \), leading to two possible energy solutions:
\begin{align*}
    E = +mc^2 \quad \text{and} \quad E = -mc^2.
\end{align*}
If we look at the positive energy solution \( E = +mc^2 \), the corresponding spinor \( u_+ \) can be written as:
\begin{align*}
    u_+ = \begin{pmatrix}
        u_A \\
        0
    \end{pmatrix}=\begin{pmatrix}
        \chi_1 \\
        \chi_2 \\
        0 \\
        0
    \end{pmatrix}=\phi_1 e_1 + \phi_2 e_2
\end{align*}
where for the negative energy solution \( E = -mc^2 \), the spinor \( u_- \) is given by:
\begin{align*}
    u_- = \begin{pmatrix}
        0 \\
        u_B
    \end{pmatrix}=\begin{pmatrix}
        0 \\
        0 \\
        \chi_3 \\
        \chi_4
    \end{pmatrix}=\phi_3 e_3 + \phi_4 e_4.
\end{align*}
If we insert these spinors back into the plane wave solution, we get the full wave functions for the positive and negative energy solutions:
\begin{align*}
    \psi_+(\mathbf{r}, t) &= u_+ e^{\frac{i}{\hbar}(-mc^2 t)} = \begin{pmatrix}
        \chi_1 \\
        \chi_2 \\
        0 \\
        0
    \end{pmatrix} e^{\frac{i}{\hbar}(-mc^2 t)} \\
    \psi_-(\mathbf{r}, t) &= u_- e^{\frac{i}{\hbar}(+mc^2 t)} = \begin{pmatrix}
        0 \\
        0 \\
        \chi_3 \\
        \chi_4
    \end{pmatrix} e^{\frac{i}{\hbar}(+mc^2 t)}.
\end{align*}
Where $\psi_+$ and $\psi_-$ span our four-dimensional spinor space, respectively. The components $\chi_1, \chi_2$ correspond to the two spin states of the particle with positive energy,
while $\chi_3, \chi_4$ correspond to the two spin states of the particle with negative energy. So we now have 4 independent solutions to the Dirac eqn. for a free particle at rest. 
We need the negative solutions to have a complete set of solutions for the Dirac eqn., even though they pose interpretational challenges. This is because Dirac eqn. needs four solutions. 
If we then consider the probabitliy density for these solutions, we find that both positive and negative energy solutions yield positive definite probability densities:
\begin{align*}
    \rho_+ &= \psi_+^{\dagger} \psi_+ = |\chi_1|^2 + |\chi_2|^2 \geq 0 \\
    \rho_- &= \psi_-^{\dagger} \psi_- = |\chi_3|^2 + |\chi_4|^2 \geq 0.
\end{align*}
This shows that the Dirac equation provides a consistent framework. It yieldt three properties: 
\begin{itemize}
    \item A positive definite probability density for both positive and negative energy solutions.
    \item A complete set of four independent solutions for a free particle at rest, accounting for both spin states and energy signs.
    \item The probability density is constant in time 
\end{itemize}

We now go back to non-zero momentum solutions. The general solution for a free particle with momentum \( \mathbf{p} \) can be written as:
Is our plane wave still a valid solution for non-zero momentum? Yes it is. 
We start by looking at the time independent Dirac eqn. in momentum space:
\begin{align*}
    E u(\mathbf{p}) = (c \alpha \cdot \mathbf{p} + \beta mc^2) u(\mathbf{p})
\end{align*}
We want to find the explicit form of the spinor \( u(\mathbf{p}) \) for a particle with momentum \( \mathbf{p} \). So 
we start by choosing $\alpha$ and $\beta$ to be the representations given earlier. This gives us:
\begin{align*}
    E \begin{pmatrix}
        u_A(\mathbf{p}) \\
        u_B(\mathbf{p})
    \end{pmatrix} = \begin{pmatrix}
        mc^2 I & c \vec{\sigma} \cdot \mathbf{p} \\
        c \vec{\sigma} \cdot \mathbf{p} & -mc^2 I
    \end{pmatrix} \begin{pmatrix}
        u_A(\mathbf{p}) \\
        u_B(\mathbf{p})
    \end{pmatrix}.
\end{align*}
This leads to a set of coupled equations for the components \( u_A(\mathbf{p}) \) and \( u_B(\mathbf{p}) \):
\begin{align*}
    (E - mc^2) u_A(\mathbf{p}) &= c \vec{\sigma} \cdot \mathbf{p} \, u_B(\mathbf{p}) \\
    (E + mc^2) u_B(\mathbf{p}) &= c \vec{\sigma} \cdot \mathbf{p} \, u_A(\mathbf{p}).
\end{align*}
From this we can express \( u_B(\mathbf{p}) \) in terms of \( u_A(\mathbf{p}) \):
\begin{align*}
    u_B(\mathbf{p}) = \frac{c \vec{\sigma} \cdot \mathbf{p}}{E + mc^2} u_A(\mathbf{p}).
\end{align*}
Now we can choose \( u_A(\mathbf{p}) \) to be an arbitrary two-component spinor, which we can denote as \( \chi_s \), where \( s \) labels the spin state (e.g., spin up or spin down).
Thus, the full spinor \( u(\mathbf{p}) \) can be written as:
\begin{align*}
    u(\mathbf{p}) = \begin{pmatrix}
        \chi_s \\
        \frac{c \vec{\sigma} \cdot \mathbf{p}}{E + mc^2} \chi_s
    \end{pmatrix}.
\end{align*}
These spinors \( u(\mathbf{p}) \) represent the positive energy solutions of the Dirac equation for a free particle with momentum \( \mathbf{p} \).
where we end up with the two possible solutions for each momentum state, corresponding to the two spin states of the particle.
\begin{align*}
    u(\mathbf{p})_1 = \begin{pmatrix}
    1\\ 0 \\ \frac{cp_z}{E+mc^2} \\ \frac{c(p_x + i p_y)}{E+mc^2}
    \end{pmatrix}, \quad 
    u(\mathbf{p})_2 = \begin{pmatrix}
    0\\ 1  \\ \frac{c(p_x - i p_y)}{E+mc^2} \\ \frac{-cp_z}{E+mc^2} 
    \end{pmatrix}.
\end{align*}
If we then instead look at the negative energy solutions, we can follow a similar procedure. 
Defining $u_B$ in terms of $u_A$ we get:
\begin{align*}
    u_A(\mathbf{p}) = \frac{c \vec{\sigma} \cdot \mathbf{p}}{E - mc^2} u_B(\mathbf{p}).
\end{align*}
Choosing \( u_B(\mathbf{p}) \) to be an arbitrary two-component spinor \( \chi_s \), we can write the full spinor \( u(\mathbf{p}) \) for negative energy solutions as:
\begin{align*}
    u(\mathbf{p})_3 = \begin{pmatrix}
    \frac{cp_z}{E-mc^2}\\ \frac{c(p_x + i p_y)}{E-mc^2} \\ 1 \\ 0
    \end{pmatrix}, \quad 
    u(\mathbf{p})_4 = \begin{pmatrix}
    \frac{c(p_x - i p_y)}{E-mc^2} \\ \frac{-cp_z}{E-mc^2} \\0\\ 1 
    \end{pmatrix}.
\end{align*}
So we get that the plane wave solutions for a free particle Dirac eqn. have the form 
\begin{align*}
    \psi(\mathbf{r}, t) = u_i(E,\mathbf{p}) e^{\frac{i}{\hbar}(\mathbf{p} \cdot \mathbf{x} - Et)}, \quad i=1,2,3,4
\end{align*}
These satisfy the Dirac equation for a free particle with momentum \( \mathbf{p} \) and energy \( E \), provided that the 
energy and momentum satisfy the relativistic energy-momentum relation $E^2 = p^2 c^2 + m^2 c^4$. 
If we use this solution for $\mathbf{p}=0$ we see that the eigenvalues are $E = \pm mc^2 $ which is the rest 
frame solution we found earlier.

\textcolor{red}{Read sec. 8.2.2 in Sakurai for another approach. Helicity, chirality, L and R handeness spinors.}

\underline{Physical interpretation:}

As described under the KG eqn., the negative energy solutions pose interpretational challenges.
It was first suggested that these negative energy states were protons and electrons where the negative energy states. 
But this violates charge conservation, as electrons and protons have opposite charges.
Dirac then proposed the hole theory, which suggested that all negative energy states are filled in the vacuum, forming a "sea" of negative energy electrons.
This solution made the vacuum stable, as all negative energy states were filled, preventing electrons from decaying into them and thereby letting 
the lowest energy state be the vacuum itself as all negative energy states are filled and all positive energy states empty.


If an electron were removed from the sea of neagtive energy states, it would leave behind a "hole" that would 
behave like a particle with opposite charge and momentum but positive energy. This hole was interpreted as a positron (same idea as a proton but same mass as an electron), the antiparticle of the electron.
This led to the definition of antimatter, where every particle has a corresponding antiparticle with the same mass but opposite charge and quantum numbers.


So if a electron in the negative energy sea gains enough energy (e.g., from a photon), it can be excited to a positive energy state, leaving behind a hole in the negative energy sea, called a positron. 
This reaction is known as pair production, where a photon can create an electron-positron pair.'
If the electron then falls back into the hole, the electron and positron annihilate each other (pair annihilation), releasing energy in the form of photons.

The interpretation was later reformulated in the context of quantum field theory (QFT), where particles and antiparticles are
 treated as excitations of underlying fields. Here the negative energy solutions were interpreted as negative energy particles that travels 
 backwards in time, which is equivalent to positive energy antiparticles traveling forward in time. This interpretation avoids the need for the Dirac sea and provides a more consistent framework for understanding particle-antiparticle creation and annihilation processes.

This also makes it more convinient to write $E\rightarrow -E, \mathbf{p} \rightarrow - \mathbf{p}$ and then 
write $v_2(E, \mathbf{p
}) = u_3(-E, -\mathbf{p
})$ and $v_1(E, \mathbf{p
}) = u_4(-E, -\mathbf{p
})$


Dirac's eqn. also successfully predicted the intrinsic spin-1/2 of the electron and its magnetic moment, which were later confirmed experimentally.
This can be seen if we find the expectation value of the hamiltonian and the orbital angular momentum operator. In non-relativistic QM the two operators commute,
but in relativistic QM they do not commute. This means that the total angular momentum must include an additional term, which is identified as the spin angular momentum.
The spin is an intrinsic form of angular momentum that is not associated with any spatial motion, but rather is a fundamental property of the particle itself, given as 
$\mathbf{S} = \frac{\hbar}{2} \Sigma$ where $\Sigma = \begin{pmatrix}
    \sigma & 0 \\
    0 & \sigma
\end{pmatrix}$.
This therefore explains why there are two solutions for each energy state, corresponding to the two possible spin states of the electron (spin up and spin down).

% https://www.youtube.com/watch?v=KdEZ-GOgZ3s&list=PLb8Y66BCLlbwXNh26fP_dIZ5DiDXUbROt&index=15



