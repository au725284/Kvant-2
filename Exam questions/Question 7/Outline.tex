\textbf{Describe the concept and formalism of discrete symmetries in
quantum mechanics, in particular, time reversal symmetry.}


\underline{Symmetries in QM}

\begin{itemize}
    \item Transformation which leaves the physical properties of a system invariant.
    \item Examples: spatial translations, rotations, time reversal.
    \item The symmetry operator must satisfy Wigner's theorem: it is either unitary or antiunitary.
    \item Antiunitary operators: 
    \begin{itemize}
        \item Antilinear: \( \hat{A} (c_1 \ket{\psi_1} + c_2 \ket{\psi_2}) = c_1^* \hat{A} \ket{\psi_1} + c_2^* \hat{A} \ket{\psi_2} \)
        \item Preserves inner products up to complex conjugation: \(\braket{\alpha|\beta} = \braket{\beta|\alpha}^*\)
        \item Can be written as the product of a unitary operator and the complex conjugation operator \( \hat{K} \): \( \hat{A}= \hat{U}\hat{K} \)
    \end{itemize}
    \item continuous symmetries can be made infinitesimal, discrete symmetries cannot.
    \item Noether's theorem only applies to continuous symmetries.
    \item Discrete symmetries lead to selection rules and degeneracies in the energy spectrum.
    \item Examples of discrete symmetries: parity (spatial inversion), time reversal, lattice translation.
\end{itemize}

\underline{Time-reversal symmetry}
\begin{itemize}
    \item Represents the reversal of the direction of time: \( t \rightarrow -t \).
    \item Time-reversal operator \( \Theta \) is antiunitary $\Rightarrow$ $\hat{H} \hat{\Theta}= \hat{\Theta} \hat{H}$
    \item Reverse momenta and angular momenta, but leave positions unchanged $\Rightarrow$ reversal of motion \textcolor{blue}{(figure)}
    \item Properties of the time-reversal operator:
    \begin{itemize}
        \item $\Theta \psi(\mathbf{r}, t) = \psi^*(\mathbf{r}, -t)$
        \item Involution: \( \Theta^2 = \pm 1 \) (the sign depends on the system)
        \item Expectation values of observables transform as:  $ \braket{\beta|\hat{X}|\alpha }= \braket{\tilde{\alpha} |\Theta \hat{X}^{\dagger} \Theta^{-1}| \tilde{\beta} }  $\newline where \( \ket{\tilde{\alpha}} = \Theta \ket{\alpha} \) and \( \ket{\tilde{\beta}} = \Theta \ket{\beta} \).
    \end{itemize}
    \item Even if \( \Theta \hat{X} \Theta^{-1} = \hat{X} \) and odd if \( \Theta \hat{X} \Theta^{-1} = -\hat{X} \)
    \item Behavior of observables under time reversal:
    \begin{itemize}
        \item Even: position and Hamiltonian (if the system is time-reversal invariant)
        \item Odd: Momentum and and angular momentum
    \end{itemize}
\end{itemize}

\underline{Example: spin-1/2 system}
\begin{itemize}
    \item Compare solution for $\hat{A}\ket{\mathbf{n}, +}$ and  $\hat{A}\ket{\mathbf{n}, -}$ $\Rightarrow$ $\hat{A}=\eta e^{-i\pi \hat{S}_y/\hbar} \hat{K}$
    \item So $\hat{K}$ does not change spin, but $\hat{U}=e^{-i\pi \hat{S}_y/\hbar}$ rotates spin by $\pi$ around y-axis.
    \item Only the operator applied twice change spin 
    \begin{align*}
        \Theta^2 = \begin{cases}
        +1, & \text{for integer spin} \\
        -1, & \text{for half-integer spin}
        \end{cases}. 
    \end{align*}
\end{itemize}

