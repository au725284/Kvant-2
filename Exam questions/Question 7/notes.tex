\textbf{Symmetries in QM:}

Symmetries in QM is a transformation that does not change the physical properties of a system but may change the state of the system.
Meaning the physical predictions are unchanged under the transformation: the probabilities and expectation values of observables remain the same.
In QM this means that the Hamiltonian is invariant under the symmetry transformation. Symmetries in QM can be used to define selection rules, conservation laws, and degeneracies in the energy spectrum.
 For any unitary operator \( U \) representing a symmetry, we have:
\begin{align*}
    U H U^\dagger = H\quad \Rightarrow \quad [U, H] = 0\quad \xrightarrow{\text{Heisenberg's eqn. of motion}} \quad \frac{dU}{dt} = 0
\end{align*}
This means that for a symmetry transformation to be a symmetry of the system, the operator 
\( U \) must commute with the Hamiltonian \( H \) and be a unitary or antiunitary operator.
\begin{itemize}
    \item Unitary: $U^{\dagger} = U^{-1}$ and linear: $$U(c_1 \psi_1 + c_2 \psi_2) = c_1 U\psi_1 + c_2 U\psi_2$$
    \item Antiunitary: \( AiA^{-1} = -i \) and nonlinear: $$A(c_1 \psi_1 + c_2 \psi_2) = c_1^* A\psi_1 + c_2^* A\psi_2$$
\end{itemize}
An antiunitary operator preserves probabilities but complex-conjugates amplitudes; it is antilinear and is required to represent symmetries like time reversal.
Both unitary and antiunitary operators preserve inner products, ensuring that physical predictions remain unchanged. 
To prove these properties for the antiunitary operator we start with the condition that the probabilities are preserved:'
\begin{align*}
    |\braket{\tilde{\alpha}|\tilde{\beta}}|= |\braket{\alpha|\beta}|.
\end{align*}
This condition also hold for the complex conjugate of the state. By use of this we define the antiunitary operator as
\begin{align*}
    \braket{\tilde{\alpha}|\tilde{\beta}} = \braket{\alpha|\beta}^*.
\end{align*}
From this we can show that the operator is antilinear:
\begin{align*}
    \hat{A}(c_1 \ket{\beta_1} + c_2 \ket{\beta_2}) &= \hat{U} \hat{K} (c_1 \ket{\beta_1} + c_2 \ket{\beta_2}) \\
    &= \hat{U} (c_1^* \hat{K} \ket{\beta_1} + c_2^* \hat{K}\ket{\beta_2}) \\
    &= c_1^* \hat{U} \hat{K}\ket{\beta_1} + c_2^* \hat{U} \hat{K}\ket{\beta_2} \\
    &= c_1^* \hat{A}\ket{\beta_1} + c_2^* \hat{A}\ket{\beta_2}
\end{align*}
From this  we can  show that the antiunitary operator can be written as $\hat{A} = \hat{U}\hat{K} $ where
\( \hat{U} \) is a unitary operator and \( \hat{K} \) is the complex conjugation operator.
We now check that the operator preserves inner products by defining \( \ket{\tilde{\alpha}} = \hat{A} \ket{\alpha} \) and \( \ket{\tilde{\beta}} = \hat{A} \ket{\beta} \):
\begin{align*}
    \braket{\tilde{\alpha}|\tilde{\beta}} &= \bra{\alpha} \hat{A}^\dagger \hat{A} \ket{\beta} \\   
    &= \sum_{a'}\sum_{b'} \bra{\alpha} \ket{a'}\bra{a'} \hat{A}^\dagger \hat{A} \ket{b'}\bra{b'} \ket{\beta} \\
    &= \sum_{a'}\sum_{b'} \braket{\alpha|a'} \braket{a'|\hat{K}^\dagger \hat{U}^\dagger \hat{U} \hat{K}|b'} \braket{b'|\beta} \\
    &= \sum_{a'}\sum_{b'} \braket{\alpha|a'} \braket{a'|\hat{K}^\dagger \hat{K}|b'} \braket{b'|\beta} \\
    &= \sum_{a'}\sum_{b'} \braket{\alpha|a'} \braket{a'|b'} \braket{b'|\beta} \\
    &= \braket{\alpha|\beta} = \braket{\beta|\alpha}^*
\end{align*}
So the antiunitary operator preserves inner products as required by using the fact that the unitary operator satisfies
$AA^{\dagger}=UU^{\dagger}=1$ (hermitian conjugate). 
This leads to Wigner's theorem, which states that any symmetry transformation in quantum mechanics can be represented by either a unitary or an antiunitary operator.


\textit{Continous symmetry} is a symmetry transformation that can be made by an arbitrary small change. 
Noether's theorem states that for every continuous symmetry of the action of a physical system, there is a corresponding conservation law.
This means that the unitary transformation can be expanded as \( U = e^{iG\epsilon} \approx 1 + iG\epsilon + \mathcal{0}(\epsilon^2)\) for small \( \epsilon \), 
where \( G \) is the generator of the symmetry transformation. This also leads to the generator being hermitian ($G=G^{\dagger}$)
If we take a finite $\epsilon$ 
which are infitnitesimal we can write the operator as $U= \left(1 + \frac{i\epsilon}{N} G\right)^N$ for $N\rightarrow \infty$ we have $U = e^{i\epsilon G}$.
For example, translational symmetry leads to conservation of momentum, rotational symmetry leads to conservation of angular momentum, and time translation symmetry leads to conservation of energy.

\textit{Discrete symmetry} is on the other hand a symmetry transformation that cannot be made by an arbitrary small change (you either do the transformation or not!). Here the 
higher order terms in the expansion of \( U \) cannot be neglected. 
Examples of discrete symmetries include parity (spatial inversion), time reversal, and lattice translation.
These can then be divided into two classes: infinite and finite many transformations. The infinite class are the ones that can be repeated infinite times without returning to the original state, e.g., lattice translations.
The finite class are the ones that return to the original state after a finite number of applications, e.g., parity and time reversal.
So unitary operators are representations of synnetries in Hilbert space. Observables are their generators. 



\begin{figure}[H]
  \centering
  \includegraphics[width=0.5\textwidth]{Figures/disc_cont.png}
\end{figure}


\textbf{Time reversal discrete symmetry:}

The time reversal operator \( \Theta \) is an antiunitary operator that reverses the direction of time in a quantum system. 
This can be seen if we consider the operator as being unitary then we apply it to the time-dependent Schrödinger equation:
\begin{align*}
    \hat{H}\hat{\Theta} \ket{n} = - \hat{\Theta}\hat{H} \ket{n} = - E_n \hat{\Theta} \ket{n}
\end{align*}
This would lead to negative energies which is unphysical, and therefore the time reversal operator must be antiunitary.
This leads to the property of $\hat{H} \hat{\Theta}= \hat{\Theta} \hat{H}$ meaning that the Hamiltonian commutes with the time reversal operator.

To apply the time reversal operator means that if a system evolves forward in time according to the Schrödinger equation, applying the time reversal operator will make it evolve backward in time which is equivalent to the 
system going in the opposite direction (opposite motion).
\begin{figure}[H]
    \centering
    \includegraphics[width=0.5\textwidth]{Figures/time_rev.png}
\end{figure}
The time reversal operator does not change the position of particles but reverses their momenta and angular momenta. So it reverse the evolution 
instead of changing the state itself.
In addition, since the time reversal operator is antiunitary, it also complex conjugates the wavefunction.
Mathematically, the time reversal operator can be represented as:
\begin{align*}
    \Theta \psi(\mathbf{r}, t) = \psi^*(\mathbf{r}, -t)
\end{align*}
The time reversal operator has the following properties:
\begin{itemize}
    \item Antiunitary: \( \Theta i\Theta^{-1} =-i \Theta\Theta^{-1}= -i \)
    \item Involution: \( \Theta^2 = \pm 1 \) (the sign depends on the system, e.g., for spin-1/2 particles, \( \Theta^2 = -1 \))
\end{itemize}
We now want to look at how different observables behave under time reversal. We start by defining even vs. odd operators.
An operator \( \hat{X} \) is even under time reversal if \( \Theta \hat{X} \Theta^{-1} = \hat{X} \) and odd if \( \Theta \hat{X} \Theta^{-1} = -\hat{X} \).
The behavier of observables under time reversal is given by:
\begin{itemize}
    \item Position: \( \Theta \hat{x} \Theta^{-1} = \hat{x} \) (even under time reversal)
    \item Momentum: \( \Theta \hat{p}\Theta^{-1} = -\hat{p} \) (odd under time reversal)
    \item Angular momentum: \( \Theta \hat{L} \Theta^{-1} = -\hat{L} \) (odd under time reversal)
    \item Spin: \( \Theta \hat{S} \Theta^{-1} = -\hat{S} \) (odd under time reversal)
    \item Hamiltonian: \( \Theta H \Theta^{-1} = H \) (if the system is time-reversal invariant)
\end{itemize}
Expectation values of even operators remain unchanged under time reversal, while expectation values of odd operators change sign.
We can see this by considering two states \( \ket{\alpha} \) and \( \ket{\beta} \) and a linear operator \( \hat{X} \):
\begin{align*}
    \hat{X}^{\dagger}\ket{\beta } = \ket{\gamma} \quad &\Rightarrow \quad \bra{\beta} \hat{X} = \bra{\gamma} \\
    \braket{\beta|\hat{X}|\alpha } = \braket{\gamma|\alpha} \quad &\Rightarrow \quad \braket{\tilde{\alpha} |\Theta \hat{X}^{\dagger} \Theta^{-1}| \tilde{\beta} } 
\end{align*}
where we have defined \( \ket{\tilde{\alpha}} = \Theta \ket{\alpha} \) and \( \ket{\tilde{\beta}} = \Theta \ket{\beta} \).
This leads to the general relation for Expectation values (observables):  \[ \braket{\beta|\hat{X}|\alpha }= \braket{\tilde{\alpha} |\Theta \hat{X}^{\dagger} \Theta^{-1}| \tilde{\beta} }  \].



\underline{Time-reversal for a spin-1/2 system:}

We start with the state 
\begin{align*}
    \ket{\mathbf{n}, +} = e^{-i \hat{S}_z \alpha/\hbar} e^{-i \hat{S}_y \beta/\hbar} \ket{+}
\end{align*}
We know that when we apply time reversal on this state we get  
\begin{align*}
    \Theta \ket{\mathbf{n}, +} = e^{-i \hat{S}_z \alpha/\hbar} e^{-i \hat{S}_y \beta/\hbar} \Theta \ket{+}
\end{align*}
since the time reversal operator changes the sign of spin and $i$. This means that we can write
\begin{align*}
    \Theta \ket{\mathbf{n}, +} = e^{-i \hat{S}_z \alpha/\hbar} e^{-i \hat{S}_y \beta/\hbar} \ket{-} =\eta \ket{\mathbf{n}, -}.
\end{align*}
If we compare this to the situation for $\ket{\mathbf{n}, -}$ we get
\begin{align*}
    \Theta  = \eta e^{-i \pi \hat{S}_y /\hbar} K,
\end{align*}
where we have used the fact that $\Theta$ is antiunitary and $K$ is the complex conjugation operator.
$\eta$ is an arbitrary phase factor with $|\eta|=1$.
By applying $\Theta$ twice we get
\begin{align*}
    \Theta^2 = \begin{cases}
    +1, & \text{for integer spin} \\
    -1, & \text{for half-integer spin}
    \end{cases}. 
\end{align*}


\textbf{Parity:}

Also called space inversion, is a discrete symmetry that involves the inversion of spatial coordinates.
The parity operator \( \pi \) is a unitary operator that transforms the position vector \( \mathbf{r} \) to its negative:
\begin{align*}
    \pi \psi(\mathbf{r}) = \psi(-\mathbf{r})
\end{align*}

% https://www.youtube.com/watch?v=OsvXeTEQxyg

even and odd: % https://www.youtube.com/watch?v=o8l6Yz9EHps
