\textbf{Partial wave/phase shift/spherically symmetric pot}

We want to describe the scattering of a particle by a spherically symmetric potential $V(r)$. This 
means that we want to describe the WF of the incomming and outgoing scattered wave, and to find the scattering amplitude $f(\theta, \phi)$.
To do this we use the method of partial waves and phase shifts.


We consider a spherically symmetric potential $V(r)$, where $r$ is the distance from the scattering center.
This means that the angular momentum is conserved, and we can decompose the wavefunction 
into partial waves characterized by the angular momentum quantum number $l$. This is possible because the 
potential does not depend on the angles $\theta$ and $\phi$ which leads the partial waves not to mix with each other.
In spherical symmetric potentials, the wavefunction can be expressed as a sum over spherical harmonics $Y_{lm}(\theta, \phi)$ and
radial functions $u_{l}(r)$:
\begin{align*}
    \psi_{E,l,m}(\mathbf{r}) = \frac{u_l(r)}{r} Y_{lm}(\theta, \phi).
\end{align*}
We now want to solve the time-independent Schrödinger equation in spherical coordinates:
\begin{align*}
    -\frac{\hbar^2}{2m} \nabla^2 \psi(\mathbf{r}) + V(r) \psi(\mathbf{r}) = E \psi(\mathbf{r}).
\end{align*}
There are many solutions to this eqn. depending on $E,l,m$ values. The $\ket{E,l,m}$ are eigenstates of $\mathbf{L}^2, L_z$ and $H_0$ which all commute with each other.
We want the solution for plane wave with initial condition $u_l(r=0)=0$ and energy $E= \frac{\hbar^2 k^2}{2m_0 }$. 
This is the simplest example since we get rid of the time dependence and we can focus on the spatial part of the wavefunction.
This means that the energy is constant while comming in and out of the potential, since the potential does not change the energy.
We choose to let the incomming plane wave propagate in the z-direction:
\begin{align}\label{planewaveexpansion}
   \braket{\mathbf{x}| \psi_{\text{in}}(\mathbf{r})} = e^{ikz} =\frac{1}{(2\pi)^{3/2}} \sum_{l=0}^{\infty} i^l (2l+1) A_l(r) P_l(\cos \theta),
\end{align}
where $A_l(r) = u_l{r}/r$ and reduces to spherical Bessel functions $j_l(kr)$ in the free case ($V=0$). This WF is outside the range of the potential. Here,
$P_l(\cos \theta)$ are the Legendre polynomials. We choose $u_l(r)$ such that it satisfies the radial Schrödinger equation:
\begin{align}\label{radialSchroedinger}
    \frac{d^2 u_l(r)}{dr^2} + \left(k^2-\frac{2m}{\hbar^2} V - \frac{l(l+1)}{r^2} \right) u_l(r) =0.
\end{align}
In order to determine the scattering amplitude $f(\theta)$, and the radial wavefunction $A_l(r)$, we need to analyze the asymptotic behavior of the wavefunction as $r \to \infty$.
We therefore rewrite the wavefunction for the scattering case:
\begin{align*}
    \braket{\mathbf{x}|\psi^{(\pm)}} \approx \frac{1}{L^{3/2}} \left(e^{i\mathbf{k}\cdot \mathbf{x}} + \frac{e^{\pm ikr}}{r}f(\mathbf{k}', \mathbf{k})\right)
\end{align*}
where the first term represents the incoming plane wave and the second term represents the scattered wave.
This can then be expanded in partial waves:
\begin{align}\label{partialwaveexpansion}
    \braket{\mathbf{x}|\psi^{(\pm)}} = \frac{1}{(2\pi)^{3/2}} \sum_{l=0}^{\infty}  (2l+1) \frac{P_l(\cos(\theta))}{2ik} \left(\frac{e^{\pm ikr}}{r} (1+2ikf_l(k)) - \frac{e^{-i(kr-l\pm)}}{r}\right),
\end{align}
where $f_l(k)$ is the partial wave scattering amplitude for the $l$'th partial wave.
By first solving  eqn. \eqref{radialSchroedinger} for $u_l(r)$  and thereby determine $A_l(r)$ and 
compare eqn. \eqref{planewaveexpansion} and \eqref{partialwaveexpansion}, we can determine $f_l(k)$.

We now define the phase shift $\delta_l$ as the difference in phase between the scattered wave and the free wave at large distances.
\textcolor{red}{the outgoing wave and incomming wave have the same coefficients???} which lead us to the relation:
\begin{align*}
    1- 2ikf_l(k) = e^{2i\delta_l}, \quad \text{fase factor for outgoing wave}.
\end{align*}
This can be used to express the scattering amplitude in terms of the phase shifts $\delta_l$.
\begin{align*}
    f_l(k) &= \frac{e^{2i\delta_l}-1}{2ik} \\
    & \Downarrow \\
    f(\theta)&= \frac{1}{k}\sum_{l=0}(2l+1)e^{i\delta_l}\sin(\delta_l) P_l(\cos(\theta)).
\end{align*}
This expression rests on the principles probability conservation and rotational invariance.
The phase shift $\delta_l$ contains all the information about the scattering process for each partial wave.

We now look at the radial part of the WF:
\begin{align*}\label{A}
    A_l(r) = e^{i\delta_l} \left(\cos(\delta_l) j_l(kr) - \sin(\delta_l) n_l(kr)\right),
\end{align*}
with $j_l(kr)$ and $n_l(kr)$ being the spherical Bessel and Neumann functions, respectively.
We can then define $\beta_l = \left(\frac{r}{A_l} \frac{dA_l}{dr}\right)|_{r=R^+}$ we can relate this to the phase shift:
\begin{align*}
    \tan(\delta_l) = \frac{kR j_l'(kR) - \beta_l j_l(kR)}{kR n_l'(kR) - \beta_l n_l(kR)}.
\end{align*}
So in step form to find the phase shifts:
\begin{itemize}
    \item Solve the radial Schrödinger eqn. \eqref{radialSchroedinger} for $u_l(r)$ inside the potential ($r<R$) with boundary condition $u_l(0)=0$.
    \item From this determine $A_l(r)$ and $\beta_l$ at $r=R^+$.
    \item Set $\beta_l^{(\text{outside})}=\beta_l^{(\text{inside})}$ (provided the potential is finite at $r=R$).
    \item Put this into the expression for $\tan(\delta_l)$ to find the phase shifts.
    \item From this find the scattering amplitude $f(\theta)$ by comparing from the phase shifts $\delta_l$.
\end{itemize}



\textcolor{red}{read about the circle example in sec. 4.4.4 and read hard sphere example in sec. 4.4.5}

\textbf{Low energy scattering:}

We assume spherically symmetric potential, low energy of the incomming wave and being in 3D. 
In low energy scattering, the wavelength of the incident particle is much larger than the range of the potential which is finite, 
allowing us to simplify the scattering problem. The scattering amplitude can be approximated using only the
s-wave ($l=0$) contribution, as higher partial waves are suppressed at low energies (called s-scattering). This is because the centrifugal 
barrier prevents higher angular momentum states from contributing significantly to the scattering process at low energies.
You can from 
\begin{align*}
    V_{\text{eff}}(r) = V(r) + \frac{l(l+1)\hbar^2}{2mr^2},
\end{align*}
see that for $l>0$, there is a centrifugal barrier that dominates at small $r$ and $l>0$. We use this potential 
since we only consider radial SE. This is because we use partial wave formalism where the angular part is separated out since 
for a spherically symmetric potential, the angles do not affect the potential.
This can also be seen from the relation between the wave number $k$ and the angular momentum quantum number $l$:
\begin{align*}
    l\sim kb,
\end{align*}
with $b$ being the impact parameter. For low energies, $k$ is small, leading to small $l$ values. Eitherway the impact must be 
very large for $l>0$ which means that the particle does not come close to the scattering center.
If we write the integral form of the partial wave as:
\begin{align*}
    \frac{e^{i\delta_l}\sin(\delta_l)}{k}= -\frac{2m}{\hbar^2} \int_0^{\infty} j_l(kr) V(r) A_l(r)r^2 dr,
\end{align*}
where if $1/k>> R$ (the range of the potential), we can approximate $j_l(kr)$ for small arguments (small $\delta_l$) to vary as $k^{2l}$
and the left hand side varies as $\delta_l/k$ which leads us to the threshold behavior:
\begin{align*}
    \delta_l \sim k^{2l+1} \quad \text{for } k\to 0.
\end{align*}
This shows that for low energies ($k\to 0$) and $l>0$, the phase shift $\delta_l \to 0$ which means that waves with 
higher angular momentum do not contribute to the scattering process, since it is not affected by the potential.

\underline{barrier/well example:} 

Consider a potential barrier/well of height/depth $V_0$ and range $R$:
\begin{align*}
    V(r) = \begin{cases}
    V_0, & r<R \\
    0, & r>R
    \end{cases}
\end{align*}
for $V_0>0$ we have a barrier and for $V_0<0$ we have an attractive well.
For this potential we get the radial equation inside the potential ($r>R$):
\begin{align*}
    rA_{l=0}(r)  \propto \sin(kr ), \quad \text{with } k = \frac{\sqrt{2m(E-V_0)}}{\hbar}. 
\end{align*}
This is for $E>V_0$ for $E<V_0$ we get
\begin{align*}
    rA_{l=0}(r) \propto \sinh(\kappa r ), \quad \text{with } \kappa = \frac{\sqrt{2m(V_0-E)}}{\hbar}. 
\end{align*}
Outside the potential ($r>R$) we have:
\begin{align*}
    A_{l=0}(r) \propto \frac{\sin(kr + \delta_0)}{kr} e^{i\delta_0}.
\end{align*}
The curvature therefore changes depending on the energy compared to the potential height/depth. 
It can be pushed in ($\delta_0>0$) or out ($\delta_0<0$) depending on the sign of $V_0$ which changes the phase shift $\delta_0$.

For $V_0<0$ we see that as we increase the depth of the well, the phase shift $\delta_0$ increases and can reach values of $\pi/2, \pi, 3\pi/2, ...$
and will eventually lead to phase shift $180^{\circ}$ (with bottom of shifted wave at the position of the top of the free wave). 
\begin{figure}[H]
    \centering
    \includegraphics[width=0.6\textwidth]{Figures/phase_shift.png}
\end{figure}
In this case,  the cross section diverges to zero 
$\sigma_{l=0}=0$ when $\delta_0 = n\pi$ with $n=0,1,2,...$ which is called the Ramsauer-Townsend effect.

\textbf{Scattering length and bound states:}

If we now consider extremely low energy scattering $k\approx 0$. For $l=0$ and $r>R$ we have the radial wavefunction by using 
\eqref{A}:
\begin{align*}
    u_{l=0}(r) = \frac{e^{i\delta_0}}{k} \left(\cos(\delta_0) \sin(kr) - \sin(\delta_0) \cos(kr)\right).
\end{align*}
in the limit $k\to 0$ we can expand the trigonometric functions to first order:
\begin{align*}
    u_{l=0}(r) &\approx \lim_{k\to 0} e^{i\delta_0}\cos(\delta_0) \left(r+\frac{\tan(\delta_0)}{k}\right) 
    &\propto \lim_{k\to 0} \left(r+\frac{1}{k \cot(\delta_0)}\right)  \equiv r-a.
\end{align*}
This means that the outside radial wavefunction behaves linearly with $r$ slightly larger than $R$, with an offset given by the scattering length $a$.
To find the scattering length one should draw the tangent of the radial wavefunction at $r=R$ and see where it intersects the r-axis.
\begin{figure}[H]
    \centering
    \includegraphics[width=0.6\textwidth]{Figures/a.png}
    \caption{The scattering length $a$ is defined as the point where the tangent to the radial wavefunction at $r=R$ intersects the r-axis.}
\end{figure}
In the limet $k\to 0$ the total cross section is given by:
\begin{align*}
    \sigma_{l=0} = 4\pi a^2.
\end{align*}
For repulsive potentials ($V_0>0$), the scattering length is positive ($a>0$) since the radial wavefunction is pushed outwards and around the order of 
the potential range (R). For attractive potentials ($V_0<0$), the scattering length can be either 
positive or negative depending on the depth of the potential well (see image below).
\begin{figure}[H]
    \centering
    \includegraphics[width=0.6\textwidth]{Figures/a_value.png}
    \caption{The scattering length $a$ can be positive or negative for attractive potentials depending on the depth of the well.}
\end{figure}


\underline{Bound states:}

As we increase the depth of the potential well ($V_0<0$), we can reach a critical depth where a bound state forms. This occurs when the scattering length diverges to positive infinity ($a \to +\infty$).
At this point, the phase shift $\delta_0$ reaches $\pi/2$, indicating a resonance in the scattering process.
For this critical depth we reach bound state with energy $E = - \frac{\hbar^2 \kappa^2}{2m}$, and the radial wavefunction outside the potential ($r>R$) behaves as:
\begin{align*}
    u_{l=0}(r) \propto e^{-\kappa r}, \quad \text{with } \kappa = \frac{\sqrt{2m|E_b|}}{\hbar}.
\end{align*}
As $\kappa \to 0$ the WF extends to very large distances. This means that just before the formation of a bound state, the scattering length becomes very large and positive ($a \to +\infty$).
The physical meaning of the scattering length is that it characterizes the effective range of the interaction between the scattering particle and the potential.
For $E= 0^-$ we have a bound state and for $E=0^+$ we have a scattering state. 
The scattering length therefore provides insight into the nature of the interaction potential and the possibility of bound state formation.