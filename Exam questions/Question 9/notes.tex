\textbf{Basic idea of scattering in QM:}

The scattering process is described by the interaction between an incoming particle (or wave) and a target potential. 
The goal is to determine how the incoming wave is modified by the potential, leading to scattered waves.
The scattered wave can be analyzed to extract information about the potential and the nature of the interaction.

The general aim is to find the WF for the incoming + scattered wave, given an incident wave and a potential.
And to calculate the probability to detect scattered particles at different angles (scattering cross-section which depends on the 
scattering amplitude).


\begin{figure}[H]
    \centering
    \includegraphics[width=0.5\textwidth]{Figures/scattering.png}
    \caption{Scattering setup where an incoming plane wave interacts with a scattering potential, resulting in scattered waves detected at various angles.}
\end{figure}

\textcolor{red}{Classical scattering???}

\underline{Perturbation theory:}

We can treat the scattering potential as a perturbation to the free particle Hamiltonian.
The total Hamiltonian is given by:
\[H = H_0 + V\]
where \(H_0\) is the free particle Hamiltonian and \(V\) is the scattering potential. The potential is 
assumed to go to zero at large distances from the scattering center (so at the detector's place). 
We describe the incoming particle by a plane wave (more simple) or a wave packet (more realistic but harder) - this is reasonable 
if the wave packets are broad compared to the target. 
We choose to describe the Hilbert space as a cube with lenghth \(L\) and take the limit \(L \to \infty\) at the end of the calculation.
This makes it possicle to describe the states with discrete momenta, which simplifies the math.
If the potential is time independent, we can use stationary perturbation theory since the plane wave has been there forever 
(so no time dependence to the plane wave either) . 
We use the time-independent Schrödinger equation:
\begin{align*}
H \ket{\psi} &= E \ket{\psi} \\
(H_0 + V) \ket{\psi} &= E \ket{\psi} \\
(E - H_0) \ket{\psi} &= V \ket{\psi} 
\end{align*}
with $H_0 = \frac{p^2}{2m}$ (free particle). 
We only want the solutions that behave like an incoming plane wave as $V\rightarrow 0$ called $\ket{\psi^{(+)}}$.
The target is a potential and therefore does not change its energy or the energy of the incoming particle (the outgoing particle 
has same energy as incomming), so we have elastic scattering, $E = \frac{\hbar^2 k^2}{2m}=E_i$. 

We want to solve the SE and to make it easier we use the Green's function method. Here we split the solution into two parts:
the free/incident part and the scattered part (the response to the potential).
To solve this the operator $(E_i - H_0)$ must be inverted otherwise we cannot isolate $\ket{\psi^{(+)}}$.
This is done by introducing the Green's operator defined as (done to avoid singularities when deviding with $E_i -H_0$):
\[G_0^{(+)} = \frac{1}{E - H_0 + i\epsilon}\].
This is necessary since $E_i - H_0$ has eigenvalues equal to zero (when acting on free particle states with energy $E_i$ ($(E_i - H_0 \ket{k}=0)$)),
making it non-invertible. The infinitesimal positive imaginary part $i\epsilon$ ensures that the Green's function has the correct boundary conditions for outgoing waves.
Using this we can write the formal solution to the SE as:
\begin{align*}
    \ket{\psi^{(+)}} &= \ket{k} + \frac{1}{E - H_0 + i\epsilon} V \ket{\psi^{(+)}}.
\end{align*}
Where $\ket{k}$ is the incoming plane wave. 
This equation is known as the Lippmann-Schwinger equation. It expresses the total wavefunction $\ket{\psi^{(+)}}$ as the sum of the incident wave $\ket{k}$ and the scattered wave, 
which is generated by the interaction with the potential $V$.
We can expand this equation iteratively to obtain a series expansion for the wavefunction
\begin{align*}
    \ket{\psi^{(+)}} &= \ket{k} + \frac{1}{E - H_0 + i\epsilon} V \ket{k} + \frac{1}{E - H_0 + i\epsilon} V \frac{1}{E - H_0 + i\epsilon} V \ket{k} + \ldots
\end{align*}
But for small potentials we can approximate the solution by truncating (cut off) the series after the first order term, leading to the Born approximation:
We can write a more general expression for the wavefunction:
\begin{align*}
    \ket{\psi^{(\pm)}} &= \ket{k} + \frac{1}{E - H_0 \pm i\epsilon} V \ket{\psi^{(\pm)}}.
\end{align*}



We now consider the situation where the potential is time dependent, but since it has been there forever we can set $t_0 \to -\infty$. 
We know from time dependent perturbation theory that the transition amplitude from an initial state $\ket{i}$ to a final state $\ket{n}$ is given by:
\begin{align*}
    \braket{n|\mathcal{\hat{U}}_I(t,-\infty)|i }= \delta_{ni} - \frac{i}{\hbar}\sum_{m} V_{nm} \int_{-\infty}^{t} dt' e^{i\omega_{nm} t'} \bra{m} \mathcal{\hat{U}}_I(t',-\infty) \ket{i} 
\end{align*}
where $\omega_{nm} = (E_n - E_m)/\hbar$. 
We guess the solution 
\begin{align*}
    \bra{m} \mathcal{\hat{U}}_I(t',-\infty) \ket{i}  = \delta_{mi} - \frac{i}{\hbar} T_{mi} \int_{-\infty}^{t} dt' e^{i\omega_{mi} t'}.
\end{align*}
Here $T_{mi}$ is the transition operator or matrix element, which we want to find. If we set $T_{mi}$ to be time independent 
and insert this solution back into the equation for the transition amplitude, we get:
\begin{align*}
    T_{ni}= V_{ni} + \sum_{m} \frac{V_{nm} T_{mi}}{-\hbar \omega_{mi} + i\hbar\epsilon}
\end{align*}
with $\omega_{mi}= \frac{E_m - E_i}{\hbar}$. By using that $H_0 \ket{m} = E_m \ket{m}$ we can write the matrix  element as:
\begin{align*}
    \braket{n|T|i} = \braket{n|V|i} + \sum_{m} \bra{n} V \frac{1}{E_i - H_0 + i\epsilon} \ket{m} \braket{m|T|i}
\end{align*}
Where $T$ is the transition operator:
\begin{align*}
    T = V + V \frac{1}{E_i - H_0 + i\epsilon} T.
\end{align*}
We hereby see that we get rid of the time dependence and have an equation similar to the Lippmann-Schwinger equation.
If we now iterate this eqn. we get:
\begin{align*}
    T &= V + V \frac{1}{E_i - H_0 + i\epsilon} V + V \frac{1}{E_i - H_0 + i\epsilon} V \frac{1}{E_i - H_0 + i\epsilon} V + \ldots
\end{align*}
For which it is clear that $T\ket{k} = V\ket{\psi^{(+)}}$. 



\underline{Transition rates and cross-section:}

Transition rate is deined as the probability per unit time for a transition from an initial state $\ket{i}$ to a final state $\ket{f}$.
\begin{align*}
    w_{i \to n} &= \frac{d}{dt} |\braket{n|\mathcal{\hat{U}}_I(t,t_0)|i}|^2\quad \text{for } \ket{n} \neq \ket{i}\\
    &= \frac{1}{\hbar^2} |\braket{n|T|i}|^2 \frac{2\epsilon e^{2\epsilon t}}{\omega_{ni}^2 \epsilon^2 }.
\end{align*}
We then take the limit $\epsilon \to 0$ and use that:
\[\lim_{\epsilon \to 0} \frac{\epsilon}{\omega^2 + \epsilon^2} = \pi \delta(\omega)\]
to get Fermi's golden rule:
\begin{align*}
    w_{i \to n} &= \frac{2\pi}{\hbar} |\braket{n|T|i}|^2 \delta(E_n - E_i).
\end{align*}
This can also be written as:
\begin{align*}
    w_{i \to n} = \frac{mkL^3}{(2\pi^2)\hbar^3} |\braket{n|T|i}|^2 d\Omega.
\end{align*}
This gives the transition rate from an initial state $\ket{i}$ to a final state $\ket{n}$ due to the scattering potential $V$.
Where we have used that the density of states in momentum space is given by $\frac{L^3 k^2 dk d\Omega}{(2\pi)^3}$ and that $E_n = \frac{\hbar^2 k^2}{2m}$. So we consider 
a final state in a solid angle $d\Omega$ around the direction of $\vec{k}$.
The differential scattering cross-section is defined as the ratio of the transition rate to the incident flux
\begin{align*}
    \frac{d\sigma}{d\Omega} = \frac{w_{i \to n}}{\text{incident flux}} =\left(\frac{mL^3}{2\pi \hbar^2}\right)^2 |\braket{n|T|i}|^2.
\end{align*}
The transfer rate/fastness thus depends on the matrix element of the transition operator between the initial and final states. 

\underline{Scattering amplitude:}

This is the probability amplitude for a particle to scatter into a specific direction.

So we want to the WF in position space $\braket{\mathbf{x}|\psi^{(\pm)}}$:
\begin{align*}
    \psi^{(\pm)}(\mathbf{x}) &= \braket{\mathbf{x}|\psi^{(\pm)}} \\
    &= \braket{\mathbf{x}|k} + \bra{\mathbf{x}} \frac{1}{E - H_0 \pm i\epsilon} V \ket{\psi^{(\pm)}}
\end{align*}
We now look at the Green's function in position space:
\begin{align*}
    G^{(\pm)}(\mathbf{x},\mathbf{x}') &= \frac{\hbar^2}{2m}\bra{\mathbf{x}} \frac{1}{E - H_0 \pm i\epsilon} \ket{\mathbf{x}'} \\
    &= \frac{\hbar^2}{2m} \int \frac{d^3 k'}{(2\pi)^3} \frac{e^{i\mathbf{k}' \cdot (\mathbf{x} - \mathbf{x}')}}{E - \frac{\hbar^2 k'^2}{2m} \pm i\epsilon} \\
\end{align*}
Here we have used the momentum eigenstates to insert a complete set of states. and then let $L\to \infty$ to convert the sum to an integral and used the 
plane wave representation of the momentum eigenstates in position space $\braket{\mathbf{k}|\mathbf{x}}=\frac{1}{L^{3/2}}e^{-ikx}$.
We then shift to spherical coordinates in momentum space with the z-axis along $\mathbf{x} - \mathbf{x}'$ to evaluate the integral and use the redidue theorem to get:
\begin{align*}
    G^{(\pm)}(\mathbf{x},\mathbf{x}') = -\frac{1}{4\pi} \frac{e^{\pm ik|\mathbf{x} - \mathbf{x}'|}}{|\mathbf{x} - \mathbf{x}'|}    
\end{align*}
with $k = \sqrt{2mE}/\hbar$.

We can now write the wavefunction in position space as:
\begin{align*}
     \braket{\mathbf{x}|\psi^{(\pm)}} =  \braket{\mathbf{x}|\mathbf{k}} - \frac{2m}{\hbar^2} \int d^3 x' \frac{e^{\pm ik|\mathbf{x} - \mathbf{x}'|}}{4\pi|\mathbf{x} - \mathbf{x}'|}  \braket{\mathbf{x}'|V|\psi^{(\pm)}}.
\end{align*}
We only consider positions at large distances from the scattering center, $r = |\mathbf{x}| \to \infty$ (detectors position).
Using that $|\mathbf{x} - \mathbf{x}'| \approx r - \hat{\mathbf{r}} \cdot \mathbf{x}'$ for $r \gg |\mathbf{x}'|$ we get:
We also assume that the potential is localized around the origin, so $\braket{\mathbf{x}'|V|\psi^{(\pm)}} = V(\mathbf{x}' ) \delta^{(3)} (\mathbf{x}'- \psi^{(\pm)})$ is only non-zero for small $|\mathbf{x}'|$.
We then make a Taylor expansion around $r'/r$ to get:
\begin{align*}
    e^{\pm ik|\mathbf{x} - \mathbf{x}'|} \approx e^{\pm ikr} e^{\mp ik \hat{\mathbf{r}} \cdot \mathbf{x}'}.
\end{align*}
Then for large $r$ we have:
\begin{align*}
    \braket{\mathbf{x}|\psi^{(\pm)}} \approx \frac{1}{L^{3/2}} \left(e^{i\mathbf{k}\cdot \mathbf{x}} + \frac{e^{\pm ikr}}{r}f(\mathbf{k}', \mathbf{k})\right)
\end{align*}
with the scattering amplitude defined as:
\begin{align*}
    f(\mathbf{k}', \mathbf{k}) = - \frac{mL^3}{2\pi \hbar^2} \braket{\pm\mathbf{k}'|V|\psi^{(\pm)}}. 
\end{align*}
From this it can be seen that the incomming plane wave is described by $e^{i\mathbf{k}\cdot \mathbf{x}}$ in the direction of $\mathbf{k}$ 
and the scattered wave (outgoing) is described by the spherical wave $\frac{e^{\pm ikr}}{r}f(\mathbf{k}', \mathbf{k})$ in the direction of $\mathbf{k}'$.
The $\pm$ sign corresponds to outgoing (+) and incoming (-) spherical waves so you can say that the negative sign corresponds to 
time reversal of the positive sign (process running backwards in time).
The differential cross-section can now be expressed in terms of the scattering amplitude as:
\begin{align*}
    \frac{d\sigma}{d\Omega} = |f(\mathbf{k}', \mathbf{k})|^2.
\end{align*}



\textbf{The Born Approximation:}

To calculate the scattering amplitude ($\braket{\mathbf{k}'|V|\psi^{(\pm)}} = \braket{\mathbf{k}'|T|\mathbf{k}} $) we need to know the full wavefunction $\ket{\psi^{(\pm)}}$ or $T$. 
This is however complicated, so we use the Born approximation to simplify the calculation.
We therefore truncate the T-operator expansion after the first term:
\begin{align*}
    T \approx V.
\end{align*}
This gives the first Born approximation for the scattering amplitude:
\begin{align*}
    f(\mathbf{k}', \mathbf{k}) \approx - \frac{m}{2\pi \hbar^2} \int d^3 x' e^{-i\mathbf{k}' \cdot \mathbf{x}'} V(\mathbf{x}') e^{i\mathbf{k} \cdot \mathbf{x}'}.
\end{align*}
This is just the Fourier transform of the potential evaluated at the momentum transfer $\mathbf{q} = \mathbf{k} - \mathbf{k}'$. 
We now look at some specific cases.

\underline{Spherical symmetric potential:}

For a spherically symmetric potential $V(\mathbf{x}) = V(r)$ we can simplify the scattering amplitude further.
In this case the scattering amplitude only depends on the magnitude of the momentum transfer $q = |\mathbf{q}|=2k\sin\left(\frac{\theta}{2}\right)$
and thereby only depend on the scattering angle. And $\mathbf{x}=r$. So we get in the first born approximation:
\begin{align*}
    f^{(1)}(\theta) \approx - \frac{2m}{\hbar^2 q} \int_0^{\infty} r V(r) \sin(qr) dr.
\end{align*}

\textit{Finite square well} simplifies things further by considering:
\begin{align*}
    V(r) = \begin{cases}
    V_0 & r \leq a \\
    0 & r > a
    \end{cases}
\end{align*}
Then we get:
\begin{align*}
    f^{(1)}(\theta) \approx - \frac{2m V_0 a^3}{(\hbar q a)^2} \left(\frac{\sin(qa)}{qa} - \cos(qa)\right).
\end{align*}
when $f^{(1)}(\theta) =0$ we can find the radius of the well $a$ from the scattering data.



\textit{Yukawa potential} is given as 
\begin{align*}
    V(r) = -V_0 \frac{e^{-\mu r}}{\mu r}.
\end{align*}
here $V_0$ is the strength of the potential and $\mu$ is related to the range of the interaction.
The first Born approximation for the scattering amplitude becomes:
\begin{align*}
    f^{(1)}(\theta) \approx - \frac{2m V_0}{\mu \hbar^2 (\mu^2 + q^2)}.
\end{align*}
Here we have used $\sin(qr)=Im(e^{iqr})$ and $q^2=2k^2 (1-\cos(\theta))$.
This leads us to the differential cross-section:
\begin{align*}
    \frac{d\sigma}{d\Omega} \approx \left(\frac{2m V_0}{\mu \hbar^2}\right)^2 \frac{1}{(\mu^2 2k^2(1-\cos(\theta)))^2}.
\end{align*}
For $\mu \to 0$ we get the Coulomb potential $V(r) = -\frac{V_0}{r}$ and the differential cross-section becomes (setting $V_0 = ZZ'e^2$):
\begin{align*}
    \frac{d\sigma}{d\Omega} \approx \frac{(2m)^2 (ZZ'e^2)^2}{\hbar^4 (4k^2 \sin^2(\theta/2))^2}.
\end{align*}


There are some generel remarks to make if the Born approximation is valid:
\begin{itemize}
    \item The cross-section and scattering amplitude denpends only on the energy and angle and not on the sign of the potential.
    \[2k^2(1-\cos(\theta)) \quad \text{and} \quad k=\sqrt{\frac{E2m}{\hbar^2}} \]
    \item $f(\theta)$ is always real, meaning that there is no phase shift between the incoming and scattered waves.
    \item $\frac{d\sigma}{d\Omega}$  is independent of the sign of the potential, meaning that attractive and repulsive potentials produce the same scattering pattern in this approximation.
    \item For low energies ($k \to 0$), the scattering amplitude becomes 
    \[ f^{(1)}(\theta)= - \frac{1}{4\pi}\frac{2m}{\hbar^2} \int V(r) d^3x\]
    meaning we have a spherical symmetric scattering pattern (s-wave scattering).
    \item For high energies ($k \to \infty$), the scattering amplitude goes to zero since the oscillatory nature of the integrand leads to cancellations in the integral.
    Meaning that we have fast osciallations which corresponds to small wavelengths that are less affected by the potential.
\end{itemize}



\underline{Validation of Born Approximation:}
\begin{align*}
     \braket{\mathbf{x}|\psi^{(\pm)}} =  \braket{\mathbf{x}|\mathbf{k}} - \frac{2m}{\hbar^2} \int d^3 x' \frac{e^{\pm ik|\mathbf{x} - \mathbf{x}'|}}{4\pi|\mathbf{x} - \mathbf{x}'|} V(\mathbf{x}') \braket{\mathbf{x}'|\mathbf{k}}.
\end{align*}
The important assumption is $T=V$ which is valid for weak potentials where higher order terms in the T-operator expansion can be neglected.
This results in $\ket{\psi^{(\pm)}}=\ket{\mathbf{k}}$.  This means that the second term must be small compared to the first term and therefore 
is the incoming plane wave only slightly modified by the potential. 
Therefore the born approximation assumes weak scattering/potential.  
We now try to estimate the validity by considering the scattered wave at a given position compared to the potential:
If we look at $|\mathbf{x}|\to \infty$ then the scattered wave behaves as a spherical wave with amplitude $\sim 1/r$ thus constant 
and small. 
If we instead look at the scattered wave close to the scattering center where the potential is significant, we can 
approximate $|\mathbf{x} - \mathbf{x}'| \approx |\mathbf{x}'|$ for small $|\mathbf{x}|$. This spimpligfies the integration
and using the plane wave representation and angular integration we get the condition for the validity of the Born approximation:
\begin{align*}
    1 \gg |\frac{m}{i\hbar^2 k} \int_0^{\infty} (e^{2ikr'}-1) V(r') dr' |.
\end{align*}
For low energies ($k \to 0$) we taylor expand the exponential to get:
\begin{align*}
    1 \gg \frac{m|V_0|a^2}{\hbar^2}
\end{align*}
At high energies ($k \to \infty$) the oscillation runs fast and the exponential cancels out, leading to:
\begin{align*}
    1 \gg \frac{m|V_0|a}{\hbar^2 k}.
\end{align*}
where we approximate $\log(ka)\approx 1$. So for large $k$ the condition is easier to satisfy (higher energies). 
This means that if the potential is weak compared to the kinetic energy of the incoming particle, the Born approximation is valid. And if 
the potential is strong enough to develop bound states, the Born approximation will fail.

\underline{Higher order Born Approximation:}
The second Born approximation can be obtained by including the next term in the T-operator expansion:
\begin{align*}
    T \approx V + V \frac{1}{E_i - H_0 + i\epsilon} V.
\end{align*}
This leads to the second Born approximation for the scattering amplitude:
\begin{align*}
    f^{(2)}(\mathbf{k}', \mathbf{k}) = - \frac{mL^3}{2\pi \hbar^2} \left[\braket{\mathbf{k}'|V|\mathbf{k}} + \sum_{m} \frac{\bra{\mathbf{k}'} V \ket{m} \braket{m|V|\mathbf{k}}}{E_i - E_m + i\epsilon}\right].
\end{align*}
The physical interpretation of the second term is that it accounts for processes where 
the particle scatters off the potential twice before reaching the final state. This means that it interacts at 
$\mathbf{x}''$ with $V(\mathbf{x}'')$ and then propagates via Green's function to $\mathbf{x}'$ where it interacts again with $V(\mathbf{x}')$ 
before being detected with momentum $\mathbf{k}'$. So the second Born approximation includes 2 scattering events, and for higher orders more events,
providing a more accurate description of the scattering process for stronger potentials.

\begin{figure}[H]
    \centering
    \includegraphics[width=0.5\textwidth]{Figures/higher_order.png}
    \caption{Figure representing the second Born approximation, where the particle scatters twice off the potential before reaching the final state.}
\end{figure}

\textcolor{red}{\underline{Example with hard sphere poential:}
}

