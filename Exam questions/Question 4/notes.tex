\textbf{The rotation group:}

The rotation group in three dimensions is denoted as SO(3), 
which stands for the Special Orthogonal group in three dimensions.
It consists of all rotations about the origin of a three-dimensional Euclidean space 
that preserve the orientation (angle) and distances. Rotations in SO(3) works under 
the operation of composition, meaning that performing one rotation followed by another
is equivalent to a single rotation. In addition the rotation group is nonabelian (non commutive), meaning 
that the order in which rotations are performed matters (see figure \ref{non commutive rotations}). 
\begin{figure}[H]
  \centering
  \includegraphics[width=0.5\textwidth]{Figures/non.png}
  \caption{}\label{non commutive rotations}
\end{figure}
In S0(3), every rotation can be represented by a $3\times 3$ orthogonal matrix with a determinant of +1 and a unitary matrix.
\begin{align*}
    R_x(\phi) = \begin{bmatrix}
        1 & 0 & 0 \\
        0 & \cos(\phi) & -\sin(\phi) \\
        0 & \sin(\phi) & \cos(\phi)
    \end{bmatrix},
    \quad R_y(\phi) = \begin{bmatrix}
        \cos(\phi) & 0 & \sin(\phi) \\
        0 & 1 & 0 \\
        -\sin(\phi) & 0 & \cos(\phi)
    \end{bmatrix},
    \quad R_z(\phi) = \begin{bmatrix}
        \cos(\phi) & -\sin(\phi) & 0 \\
        \sin(\phi) & \cos(\phi) & 0 \\
        0 & 0 & 1
    \end{bmatrix}.
\end{align*}
These represents finite rotations about the x, y, and z axes by an angle $\phi$. 

\underline{Infinitesimal rotations:}

If we instead 
consider infinitesimal rotations, we can expand over small angles ($\epsilon$) and express them as 
\begin{align*}
    R_x(\epsilon) &\approx \begin{bmatrix}
        1 & 0 & 0 \\
        0 & 1-\frac{\epsilon^2}{2} & -\epsilon \\
        0 & \epsilon & 1-\frac{\epsilon^2}{2}
    \end{bmatrix},\quad
    R_y(\epsilon) &\approx \begin{bmatrix}
        1-\frac{\epsilon^2}{2} & 0 & \epsilon \\
        0 & 1 & 0 \\
        -\epsilon& 0 & 1-\frac{\epsilon^2}{2}
    \end{bmatrix},\quad
    R_z(\epsilon)&\approx \begin{bmatrix}
        1-\frac{\epsilon^2}{2} & -\epsilon & 0 \\
        \epsilon & 1-\frac{\epsilon^2}{2} & 0 \\
        0 & 0 & 1
    \end{bmatrix}
\end{align*}
For very small angles, we can neglect the second-order terms, 
which will change the property og non commutativity. This is seen if we look at 
\begin{align*}
    R_x(\epsilon)R_y(\epsilon) - R_y(\epsilon)R_x(\epsilon) = 
    \begin{bmatrix}
         0 & -\epsilon^2 & 0 \\ 
         \epsilon^2 & 0 & 0 \\ 
         0 & 0 & 0 
    \end{bmatrix}
\end{align*}
This can also be written as $R_x(\epsilon)R_y(\epsilon) - R_y(\epsilon)R_x(\epsilon) = R_z(\epsilon^2)- R_{any}(0)=R_z(\epsilon^2)-1$.
 

\underline{Rotations in QM:}

In quantum mechanics, rotations are represented by unitary operators that act on the state vectors in a Hilbert space.
As seen for both translations and time evolution, we can express an infinitesimal operator as 
\begin{align*}
    U(R(\epsilon)) = 1 - i G\epsilon,
\end{align*}
where $G$ is the generator of the rotation. We know from classical mechanics that angular momentum is the generator of rotations,
the same as the momentum is the generator of translations and the Hamiltonian is the generator of time evolution.
So therefore we define the angular momentum operator $\hat{J}$ as the generator of rotations, and let $\epsilon \rightarrow d\phi$ giving us
\begin{align*}
    \mathcal{D}(\vec{n}, d\phi)= 1 - i\frac{\vec{J}\cdot \vec{n}}{\hbar}  d\phi,
\end{align*}
Where $\vec{n}$ is the axis of rotation and the operator $\vec{J}$ ($\hat{J}$) is taken to be hermitian and therefore lets the rotation operator be unitary. 

If we now consider finite rotations by combining many infinitesimal rotations, we get
\begin{align*}
    \mathcal{D}(\vec{n}, \phi) = \lim_{N\to \infty} \left( 1 - i\frac{\vec{J}\cdot \vec{n}}{\hbar}  \frac{\phi}{N} \right)^N 
    = \exp\left(-i\frac{\vec{J}\cdot \vec{n}}{\hbar} \phi\right),
\end{align*}
where we have used the definition of the exponential function.
In addition we postulate that $\mathcal{D}(R)$ has the same group properties as the rotation group $R$ (see figure below)
\begin{figure}[H]
  \centering
  \begin{subfigure}{0.4\textwidth}
    \centering
    \includegraphics[width=\textwidth]{Figures/properties1.png}
  \end{subfigure}
  \hfill
  \begin{subfigure}{0.4\textwidth}
    \centering
    \includegraphics[width=\textwidth]{Figures/properties2.png}
  \end{subfigure}
\end{figure}

\underline{Example with spin-1/2:}

If we now try to investigate how the rotation operator works on a ket and thereby how it affects the physical system. 
We check this by finding out how the expectation values change under rotation, and thereby how it effects the state of the system.
We use the spin-1/2 system as an example, where we look at a rotation around the z-axis.  

We start be defining the base kets for a spin-1/2 system as $\{  \ket{+}, \ket{-}\}$. 
For the z-spin operator the eigenvalues are $\pm \hbar/2$ and the corresponding eigenkets are $\ket{+}$ and $\ket{-}$.




The angular momentum operator for spin-1/2 particles can be expressed in terms of the Pauli matrices ($\vec{J} = \frac{\hbar}{2}\vec{\sigma}$), or in terms of the spin operators ($\vec{J} = \hat{S}_K$).
The three spin operators are given by:
\begin{align*}
    \hat{S}_x = \frac{\hbar}{2} \left(\ket{+}\bra{-}+ \ket{-}\bra{+}\right), \quad
    \hat{S}_y = \frac{i\hbar}{2} \left(\ket{-}\bra{+}- \ket{+}\bra{-}\right), \quad
    \hat{S}_z = \frac{\hbar}{2} \left(\ket{+}\bra{+}- \ket{-}\bra{-}\right).
\end{align*}

We start by looking at how the rotation operator effects the expectation value: 
$\braket{S_x} \rightarrow\braket{S_x}_R = \braket{\phi | \hat{R}_z^{\dagger} \hat{S}_x \hat{R}_z | \phi}$. 
This is the operator we are interested in, so we calculate it step by step.
We start by calculating $\hat{R}_z^{\dagger} \hat{S}_x \hat{R}_z$:
\begin{align*}
    \hat{R}_z^{\dagger} \hat{S}_x \hat{R}_z = e^{i\frac{\hat{S}_z}{\hbar}\phi} \hat{S}_x e^{-i\frac{\hat{S}_z}{\hbar}\phi}.
\end{align*}
We now insert the Taylor expansion of the exponentials and the spin operator, and get
\begin{align*}
    \hat{R}_z^{\dagger} \hat{S}_x \hat{R}_z &= e^{i\frac{\hat{S}_z}{\hbar}\phi} \frac{\hbar}{2} \left(\ket{+}\bra{-}+ \ket{-}\bra{+}\right) e^{-i\frac{\hat{S}_z}{\hbar}\phi}\\
    &= \frac{\hbar}{2} \left( e^{i\frac{\phi}{2}} \ket{+}\bra{-} e^{i\frac{\phi}{2}} + e^{-i\frac{\phi}{2}} \ket{-}\bra{+} e^{-i\frac{\phi}{2}} \right)\\
    &= \frac{\hbar}{2} \left( e^{i(\phi/2+\phi/2)} \ket{+}\bra{-} + e^{-i(\phi/2 + \phi/2)} \ket{-}\bra{+} \right).
\end{align*}
Where the last equal sign holds because the exponentials are phase factors or just complex numbers. 
I know that $e^{i\phi}=\cos(\phi) + i \sin(\phi)$ and $e^{-i\phi}=\cos(\phi) - i \sin(\phi)$, so I can rewrite the expression as
\begin{align*}
    \hat{R}_z^{\dagger} \hat{S}_x \hat{R}_z &= \frac{\hbar}{2} \left( \cos(\phi) + i \sin(\phi) \right) \ket{+}\bra{-} + \frac{\hbar}{2} \left( \cos(\phi) - i \sin(\phi) \right) \ket{-}\bra{+} \\
    &= \cos(\phi) \frac{\hbar}{2} \left( \ket{+}\bra{-} + \ket{-}\bra{+} \right) - i\sin(\phi) \frac{\hbar}{2} \left( \ket{+}\bra{-} - \ket{-}\bra{+} \right)\\
    &= \cos(\phi) \hat{S}_x - \sin(\phi) \hat{S}_y.
\end{align*}
So we see that if we rotate the state of the system around the z-axis a finite amount of $\phi$ the expectation
values of the observables changes as 
\begin{align*}
    \braket{S_x}_R = \braket{\phi | \hat{R}_z^{\dagger} \hat{S}_x \hat{R}_z | \phi} = \cos(\phi) \braket{S_x} - \sin(\phi) \braket{S_y}.\\
    \braket{S_y}_R = \braket{\phi | \hat{R}_z^{\dagger} \hat{S}_y \hat{R}_z | \phi} = \sin(\phi) \braket{S_x} + \cos(\phi) \braket{S_y}.\\
    \braket{S_z}_R = \braket{\phi | \hat{R}_z^{\dagger} \hat{S}_z \hat{R}_z | \phi} = \braket{S_z}.
\end{align*}

We now look at how the state ket itself changes under the rotation.
We start by looking at the inital state ket: $\ket{\psi} = \hat{I}\ket{\psi} = \ket{+}\bra{+} \ket{\psi} + \ket{-}\bra{-} \ket{\psi}$. When 
we apply the rotation operator, we get
\begin{align*}
    \ket{\phi}_R = e^{-i\frac{\phi}{2}} \ket{+}\bra{+} \ket{\psi} + e^{i\frac{\phi}{2}} \ket{-}\bra{-} \ket{\psi}.
\end{align*}
If we take the special case of $\phi = 2\phi$ we get:
\begin{align*}
    \ket{\phi}_{R_{z}(2\phi)} = e^{-i\frac{2\pi}{2}} \ket{+}\bra{+} \ket{\psi} + e^{i\frac{2\pi}{2}} \ket{-}\bra{-} \ket{\psi} = - \ket{\psi}.
\end{align*}
So we see that a rotation of $2\pi$ changes the state ket by a global phase factor of -1, and it therefore do not behave like a classical vector.

\textcolor{red}{Maybe include how the rotation operator looks for pauli matrices and maybe talk about SU2}
% https://www.youtube.com/watch?v=J5RNZ1kT1bg



\textbf{Angular momentum:}

If we go back and look at the commutation relations for the rotation operators, we see that they are given by
$R_x(\epsilon)R_y(\epsilon) - R_y(\epsilon)R_x(\epsilon) = R_z(\epsilon^2)- R_{any}(0)=R_z(\epsilon^2)-1$.
If this is now rewritten in terms of the generators of the rotations, by using the expansion of $\vec{R}= \exp\left(-i\frac{\vec{J}\cdot \vec{n}}{\hbar} \phi\right)$, we get
\begin{align*}
    [J_x, J_y] &= i \hbar J_z, \quad [J_y, J_z] = i \hbar J_x, \quad [J_z, J_x] = i \hbar J_y\\
    [J_i, J_j] &= i \hbar \epsilon_{ijk} J_k. 
\end{align*}
This is the fundamental commutation relations for angular momentum operators in quantum mechanics.
We now define the new opetator $\vec{J}^2 = J_xJ_x + J_yJ_y + J_z J_z$, which commutes with all the components of $\vec{J}$:
\begin{align*}
    [\vec{J}^2, J_i] = 0 \quad \text{for } i = x, y, z. 
\end{align*}
We now want to find the eigenvalues and eigenkets of both $\vec{J}^2$ and one of its components, usually chosen to be $J_z$.
We start by defining the eigenvalue $a$ and $b$ such that
\begin{align*}
    \vec{J}^2 \ket{a,b} = a \ket{a,b}, \quad J_z \ket{a,b} = b \ket{a,b}.
\end{align*}
To find the possible values of $a$ and $b$, we introduce the ladder operators $J_+$ and $J_-$ defined as
\begin{align*}
    J_+ = J_x + i J_y, \quad J_- = J_x - i J_y.
\end{align*}
The ladder operators have the following commutation relations with $J_z$ and $\vec{J}^2$:
\begin{align*}
    [J_z, J_{\pm}] = \pm \hbar J_{\pm}, \quad [\vec{J}^2, J_{\pm}] = 0, \quad [J_+, J_-] = 2 \hbar J_z.
\end{align*}
Using these relations, we can show that the ladder operators raise and lower the eigenvalue $b$ by $\hbar$ by looking at the action of $J_z$ on the states $J_{\pm} \ket{a,b}$:
\begin{align*}
    J_z (J_{\pm} \ket{a,b}) = (b \pm \hbar) (J_{\pm} \ket{a,b}).
\end{align*}
So the resulting ket is still an eigenket of $J_z$ with eigenvalue $b \pm \hbar$.
We now want to find the possible values of $a$ and $b$.

From the fundamental commutation relations, we can derive that the eigenvalues of $\vec{J}^2$ and $J_z$ which are given by
\begin{align*}
    a = \hbar^2 j(j+1),
    b = \hbar m,
\end{align*}
where $j$ is a non-negative integer or half-integer and $m$ takes values from $-j$ to $j$ in integer steps.
The quantum number $j$ is called the total angular momentum quantum number, and $m$ is called the magnetic quantum number.
So we get the final expressions for the eigenvalues as
\begin{align*}
    \vec{J}^2 \ket{j,m} = \hbar^2 j(j+1) \ket{j,m}, \quad J_z \ket{j,m} = \hbar m \ket{j,m}.
\end{align*}    
Since the angular momentum commutation relations are derived from the rotation properties and the generator of rotation ($J_k$) 
it all follows from the rotation group SO(3).

If we now consider the matrix element of the angular momentum operators in the basis of the eigenkets $\ket{j,m}$, we can find the matrix representation of the angular momentum operators.
The matrix elements are given by
\begin{align*}
    \bra{j,m'} J^2 \ket{j,m} = \hbar^2 j(j+1) \delta_{m',m} \delta_{j',j} \\
    \bra{j,m'} J_z \ket{j,m} = \hbar m \delta_{m',m} \delta_{j',j} \\
\end{align*}
From this we want to find the matrix elements of the ladder operators $J_+$ and $J_-$.

We start by considering $J^2 - J_z = \frac{1}{2}(J_+ J_- + J_- J_+)= \frac{1}{2}(J_+ J_+^{\dagger}+ J_+^{\dagger} J_+)$. 
So we start by looking at the matrix element $\bra{j',m'} J_+^{\dagger}J_+ \ket{j,m} =\bra{j',m'} (J^2 -J^2_z -\hbar J_z) \ket{j,m}$. 
By working this out we get the expressions
\begin{align*}
    J_+ \ket{j,m} = \hbar \sqrt{(j-m)(j+m+1)} \ket{j,m+1},\\
    J_- \ket{j,m} = \hbar \sqrt{(j+m)(j-m+1)} \ket{j,m-1}\\
    \bra{j',m'} J_{\pm} \ket{j,m} = \hbar \sqrt{(j\mp m)(j \pm m+1)} \delta_{j',j} \delta_{m',m\pm 1}. 
\end{align*}


\textcolor{red}{maybe talk about representation of the rotation operator  (3.5.4)}



