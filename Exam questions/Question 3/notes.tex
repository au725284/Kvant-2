\textbf{Propagator:}

The propagator is a function that describes/specifies the probability/transition amplitude for a particle
to travel from one point in space-time to another over a given time period.


The propagator is the transition amplitude, e.i. the probability amplitude for a particle to go from position \(x'\) at time \(t'\) to position \(x\) at time \(t\). 

probability amplitude???? greens function???

The propagator is defined/expressed as the matrix element of the time evolution operator between position eigenstates:
\begin{align*}
K(x'',t;x',t_0) &= \langle x'' | U(t,t_0) | x' \rangle.
\end{align*}

\underline{Derivation:}

We start from the time evolution operator acting on a state \(|\psi(t_0)\rangle\):
\begin{align*}
    |\psi(t)\rangle &= U(t,t_0) |\psi(t_0)\rangle.
\end{align*}
Now we want to know what the wave function is at time \(t\) in position space:
\begin{align*}
    \psi(x'',t) &= \langle x'' | \psi(t) \rangle \\
    &= \langle x'' | U(t,t_0) | \psi(t_0) \rangle
\end{align*}
Now we insert the identity operator in terms of position eigenstates:
\begin{align*}
    \psi(x'',t) &= \int dx' \langle x'' | U(t,t_0) | x' \rangle \langle x' | \psi(t_0) \rangle \\
    &= \int dx' K(x'',t;x',t_0) \psi(x',t_0).
\end{align*}
Where $\psi(x',t_0)=\langle x' | \psi(t_0) \rangle$ is the wave function at time \(t_0\) in position space. So it is a convolution of the propagator with the initial wave function.
This is unmistakably a integral operator, where the propagator acts as the kernel that evolves the wave function from time \(t_0\) to time \(t\).
So we see that the propagator \(K(x'',t;x',t_0)\) gives us the amplitude for a particle to propagate from position \(x'\) at time \(t_0\) to position \(x''\) at time \(t\).
And is given as the matrix element of the time evolution operator between position eigenstates.

So we see that the propagator is a fundamental object in quantum mechanics that encapsulates the dynamics of a quantum system and allows us to calculate how wave functions evolve over time.
This also means that if we know the propagator for a system, we can determine the time evolution of any initial wave function.
We get from this that the propagator works as the Green's function for the Schrödinger equation - it solves the equation with a delta function source term and find the dynamics of the system.


\underline{Green's Function:}

The Green function is a mathematical tool used to solve inhomogeneous differential equations. 
It is used to solve equations of the form:
\begin{align*}  
    Ly(x)=f(x)
\end{align*}
where \(L\) is a linear differential operator, \(y(x)\) is the unknown function we want to solve for, and \(f(x)\) is a known source function.
The Green's function \(G(x,x')\) is defined as the solution to the equation:
\begin{align*}
    LG(x,x')=\delta(x-x')
\end{align*}
where \(\delta(x-x')\) is the Dirac delta function.
So the propagator can be seen as the Green's function for the Schrödinger equation, allowing us to solve for the time evolution of wave functions in quantum mechanics.
So we have the integral as the L-operator.


\underline{Example of free particle:}

For a free particle, the Hamiltonian is given by:
\begin{align*}
    H = \frac{p^2}{2m}
\end{align*}
and the time evolution operator is
\begin{align*}
    U(t,t_0) = e^{-\frac{i}{\hbar}H(t-t_0)} = e^{-\frac{i}{\hbar}\frac{p^2}{2m}(t-t_0)}. 
\end{align*}
Now we can calculate the propagator for a free particle:
\begin{align*}
    K(x'',t;x',t_0) &= \langle x'' | U(t,t_0) | x' \rangle \\
    &= \langle x'' | e^{-\frac{i}{\hbar}\frac{p^2}{2m}(t-t_0)} | x' \rangle.
\end{align*}
To evaluate this, we insert a complete set of momentum eigenstates: 
\begin{align*}
    K(x'',t;x',t_0) &= \int dp \langle x'' | e^{-\frac{i}{\hbar}\frac{p^2}{2m}(t-t_0)} | p \rangle \langle p | x' \rangle \\
    &= \int dp e^{-\frac{i}{\hbar}\frac{p^2}{2m}(t-t_0)} \langle x'' | p \rangle \langle p | x' \rangle.
\end{align*}
Here the exponential can be taken out because it is just a number. This is due to the fact that the momentum eigenstates are eigenstates of the Hamiltonian for a free particle.
Therefore we can have the property $e^A \ket{a} = e^{a} \ket{a}$ ($a$ is the eigenvalue). If we then use the Taylor expansion of $e^A$ 
we see that $e^A = \sum_{n=0}^{\infty } \frac{A^n}{n!}$. So if we insert this in the eqn. above: $e^A \ket{a} = \sum_{n=0}^{\infty } \frac{A^n}{n!} \ket{a} = \sum_{n=0}^{\infty } \frac{a^n}{n!} \ket{a} = e^a \ket{a}$. 
So since $| p \rangle$ are eigenstates of the Hamiltonian, we can take the exponential out as shown above.
We therefore see that we have the plane wave representation of the position and momentum eigenstates:
\begin{align*}
    \langle x | p \rangle &= \frac{1}{\sqrt{2\pi \hbar}} e^{\frac{i}{\hbar}px}, \\
    \langle p | x \rangle &= \frac{1}{\sqrt{2\pi \hbar}} e^{-\frac{i}{\hbar}px}.
\end{align*}
Substituting these into the expression for the propagator, we get:
\begin{align*}
    K(x'',t;x',t_0) &= \int dp e^{-\frac{i}{\hbar}\frac{p^2}{2m}(t-t_0)} \frac{1}{\sqrt{2\pi \hbar}} e^{\frac{i}{\hbar}px''} \frac{1}{\sqrt{2\pi \hbar}} e^{-\frac{i}{\hbar}px'} \\
    &= \frac{1}{2\pi \hbar} \int dp e^{-\frac{i}{\hbar}\left(\frac{p^2}{2m}(t-t_0) - p(x''-x')\right)}.
\end{align*}
This integral is a Gaussian integral and can be evaluated to give:
\begin{align*}
    K(x'',t;x',t_0) &= \sqrt{\frac{m}{2\pi i \hbar (t-t_0)}} e^{\frac{i}{\hbar}\frac{m(x''-x')^2}{2(t-t_0)}}.
\end{align*}

The physical interpretation of this result is that the propagator describes how a free particle spreads out over time.
This expression shows that the propagator for a free particle has a Gaussian form, indicating that the probability amplitude for finding the particle at position \(x''\) at time \(t\) 
given it was at position \(x'\) at time \(t_0\) spreads out over time, reflecting the wave-like nature of quantum particles.




\textbf{Feynman path integrals:}

Another way to get an expression for the propagator is to use Feynman formulation of path integrals. 
First of all I will go into the idea of Feynman formulation and path integrals, before I connect it to the propagator.

The basic difference between CM and QM is that in CM we have a single, unique trajectory for a particle between two points in space-time, 
determined by the principle of least action. In QM, however, the particle does not have a single trajectory; instead,
it can take all possible paths between the two points. Therefore instead of finding the single path that the particle takes
we have to sum over all possible paths, each weighted by a phase factor determined by the action along that path ($e^{iS/\hbar}$). 
This sum is called the Feynman path integral. So in QM you calculate the probability for the particle being on a specific point 
where you in CM calculate the trajectory of the particle and at which point it will be at a specific time.
That the QM particle takes all possible paths is a consequence of the wave-particle duality.
 Here the wave aspect of the particle allows it to interfere with itself, leading to the necessity of considering all possible paths (see two-slit experiment). 

Here we can use the Heisenberg picture to express the propagator as the transition amplitude between position eigenstates:
\begin{align*}
    K(x'',t;x',t_0) &= \langle x'',t | x',t_0 \rangle.
\end{align*}
This leads us directly to the path integral formulation.

The transition amplitude is then ficen by summing over all possible paths from the initial to the final point, each contributing with a phase factor given by the action (path integral formulation):
\begin{align*}
    \langle x'',t | x',t_0 \rangle& = \sum_{paths} e^{\frac{i}{\hbar} S[path]}\\
   & = \int \mathcal{D}[x(t)] e^{\frac{i}{\hbar} S[x(t)]}
\end{align*}
Where $S(x(t))=\int_{t_0}^{t_1} dt \mathcal{L}(x, \dot{x})$ is the action along a specific path, and \(\mathcal{D}[x(t)]\) denotes the integration over all possible paths \(x(t)\)
 connecting the initial and final points ($\mathcal{D}(x(t)) = \lim_{N\rightarrow\infty} \left(\frac{m}{2\pi i\hbar \Delta t}\right)^{N/2} \prod_{j=1}^{N-1} dx_j$).
\begin{align*}
    \langle x'',t | x',t_0 \rangle& = \sum_{paths} e^{\frac{i}{\hbar} S[path]}\\
   & = \int_{x'}^{x''} \mathcal{D}[x(t)] e^{\frac{i}{\hbar} \int_{t_0}^{t} dt \mathcal{L}(x, \dot{x})}
\end{align*}
In the limit where $S>>\hbar$ (classical limit) the paths will all cancel out due to destructive interference, 
except for the path that makes the action stationary (classical path, the path with action equal to zero).
So the important point from this path integral formulation is that in the classical limit we recover the classical path of least action, 
which is consistent with the correspondence principle.

So we see that the Feynman path integral is the propagator expressed as a sum over all possible paths, each weighted by the phase factor determined by the action along that path.

\underline{Example with free particle:}

For a free particle, the Lagrangian is given by:
\begin{align*}
    \mathcal{L} = \frac{1}{2} m \dot{x}^2 .  
\end{align*}
Now we want to find an expression for $\mathcal{D}[x(t)]$, and we want to redefine this so the computation is more intuitive. 
A sum over all paths is not very practical, so we instead discretize the time interval \([t_0, t]\) into \(N\) small segments of duration \(\epsilon = \frac{t - t_0}{N}\).
At each time step, we denote the position of the particle as \(x_j\) for \(j = 0, 1, 2, \ldots, N\), where \(x_0 = x'\) and \(x_N = x''\).
By doing this we can express the path as a list of numbers instead of a continuous function ($x(t)$). 
We can approximate the transition amplitude for discrete time steps as:
\begin{align*}
    \langle x_n, t_n | x_{n-1}, t_{n-1} \rangle = \frac{1}{w(\Delta t)}e^{\frac{i}{\hbar} S(n,n-1)}.
\end{align*}
These paths are descrete and therefore it makes sense to sum over them. 
By looking at these very small time steps and thereby considerering the paths very close to the 
classical path (only the ones close to survive) we can approximate the path as a integral over these discrete positions:
\begin{align*}
    S(n,n-1) &= \int_{t_0}^{t} dt \mathcal{L}(x, \dot{x}) \\
    &= \Delta t \left( \frac{m}{2} \left( \frac{(x_n-x_{n-1})}{\Delta t}  \right)^2 - V \left( \frac{(x_n+x_{n-1})}{2}  \right)^2\right)
\end{align*}
For free particle we have no potential, so $V=0$, which simplifies the expression.
By using orthonormality in the Heisenberg picture we can find the weight factor:
\begin{align*}
    \langle x_n, t_n | x_{n-1}, t_{n-1} \rangle\big|_{t_n=t_{n-1}} &= \delta(x_n - x_{n-1}) \\
    \Rightarrow
    \langle x_n, t_n | x_{n-1}, t_{n-1} \rangle &= \sqrt{\frac{m}{2\pi i \hbar \Delta t}} e^{\frac{im(x_n-x_{n-1})^2}{2\hbar \Delta t}}.
\end{align*}

which is the same result as we obtained earlier using the propagator formalism.

